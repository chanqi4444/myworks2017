\jxhj{%教学后记
	}
\skrq{%授课日期
	2017年3月13日 4-5节}
\ktmq{%课题名称
	椭圆编程 }
\jxmb{%教学目标,每行前面要加 \item
	\item 掌握用循环来实现Z向分层;
	\item 掌握椭圆的宏程序思路;
	\item 掌握椭圆的宏程序。 }
\jxzd{%教学重点,每行前面要加 \item
	\item 循环来实现Z向分层;
	\item 椭圆的宏程序思路。 }
\jxnd{%教学难点,每行前面要加 \item
	\item 椭圆的宏程序思路。 }
\jjff{%教学方法
	通过讲述、举例、演示法来说明;}

\makeshouye %制作教案首页

%%%%教学内容
\subsection{组织教学}
\begin{enumerate}[\hspace{2em}1、]
	\item 集中学生注意力;
	\item 清查学生人数;
	\item 维持课堂纪律;
\end{enumerate}
\subsection{复习导入及主要内容}
\begin{enumerate}[\hspace{2em}1、]
\item Z向分层;
\item IF THEN 指令;
\item Z向分层的应用;
\end{enumerate}
\subsection{教学内容及过程}
\subsubsection{非圆曲线的拟合加工} %\marginpar{举例说明}
 数控机床一般中能作直线插补和圆弧插补. 遇到工件轮廓是非圆曲线的零件时. 常用直线段或圆弧去逼近非圆曲线. 即拟合加工. 只要拟合误差在允许的范围内就行了.
 
 \paragraph{分段}
 等插补段法(求最小曲率半径Rmin, 求在允许插补误差时的弦长,求坐标)
 
 等插补误差法(以起点建立误差圆, 求该圆与曲线的公切线的斜率, 以起点作公切线的平行线, 计算坐标.)
 
 其它方法: (等角度)
 
\paragraph{用直线拟合}
 弦线, 割线, 切线
 \paragraph{圆弧逼近法}
 通过三点作圆(这三点是上面分段中的三个点)
 
 计算圆心
 
 计算坐标
 
\subsubsection{椭圆的数学模型}
\paragraph {参数方程}
 $$X=A*cos(a)$$
 $$Y=B*sin(a)$$
  使用角度控制
\paragraph{ 普通方程}
 $$ \frac{X^2}{A^2} + \frac{Y^2}{B^2} =1 $$
  可使用$X$控制, 常用于车床
\subsubsection{流程控制}
\subsubsection{椭圆的宏程序}
\begin{verbatim}
N100 #3=#3-1
G1 X[#1*COS[#3]] Y[#1*SIN[#3]]
IF [#1 GT -180] GOTO100
\end{verbatim}
\subsubsection{GOTO指令}
无条件转移。
 
\subsection{课堂小结}

\begin{enumerate}[1、]
	\item 非圆曲线的拟合加工;
	\item 椭圆的数学模型;
	\item 流程控制
	\item 椭圆的宏程序
\end{enumerate}

\vfill
\subsection{布置作业}
\begin{enumerate}[1、]
	\item 编写一个比较通用的外圆加工轮廓。。 
\end{enumerate}
\vfill