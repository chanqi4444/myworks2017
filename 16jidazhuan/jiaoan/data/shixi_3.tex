\jxhj{%教学后记
	}
\skrq{%授课日期
	2017年 | 4月11日 |4月18日|4月25日  | 1-3节}
\ktmq{%课题名称
	 	工件找正、装夹与对刀、调速}
\jxmb{%教学目标,每行前面要加 \item
	\item 熟练使用百分表找正虎钳及工件;
	\item 能正确的向加工中心刀库放入刀具;
	\item 熟练安全地调节主轴转速;
\item 	正确地用“取中法”进行对刀。}
\jxzd{%教学重点,每行前面要加 \item
	\item 熟练使用百分表找正虎钳及工件;
	\item 正确地用“取中法”进行对刀。}
\jxnd{%教学难点,每行前面要加 \item
	\item 正确地用“取中法”进行对刀。}
\jjff{%教学方法
	通过讲述、举例、演示法来说明;}

\makeshouye %制作教案首页

%%%%教学内容
\subsection{实习教学要求}
\begin{compactenum}[\hspace{2em}1、]
\item 熟练使用百分表找正虎钳及工件;
\item 能正确的向加工中心刀库放入刀具;
\item 熟练安全地调节主轴转速;
\item 	正确地用“取中法”进行对刀。;
\end{compactenum}

\subsection{相关工艺}
\subsubsection{虎钳及工件的找正}
虎钳找正:清洁工作台及虎钳——安装好百分表——使百分表的表头与固定钳口接触——移动X轴——调整虎钳——固定拧紧虎钳——再打表检查。

工件找正:清洁工作台及虎钳——安装好百分表——使百分表的表头与工件接触——移动轴检查——调整工件——固定螺钉压板——再打表检查
	
找正上表面:清洁工作台及虎钳——安装好百分表——使百分表的表头与工件上表接触——移动轴检查——调整工件——固定螺钉压板——再打表检查

注意:找正时,其偏差不能超过0.01-0.02,宁紧后工件或工作台会发生移动,必须再次检查。

\subsubsection{常用夹具}

	螺钉压板:利用T形槽螺栓和压板将工件固定在机床工作台上即可。用百分表等直接找正工件

	机用虎钳:适合形状比较规则的的零件铣削,零件夹紧时要注意控制工件变形和一端钳口上翘

	铣床用卡盘:如三爪卡盘、四爪卡盘等,使用时用 T 形槽螺栓将卡盘固定在机床工作台上即可。

工件在夹具中的安装:

	使待加工面充分暴露在外,考虑机床主轴与工作台面之间的最小距离和刀具的装夹长度,确保在主轴的行程范围内能使工件的加工内容全部完成; 

	夹具在机床工作台上的安装位置必须给刀具运动轨迹留有空间,不能和各工步刀具轨迹发生干涉; 

	夹点数量及位置不能影响刚性。

\subsubsection{加工中心手动换刀过程}
确认刀具和刀柄的重量 

清洁刀柄锥面和主轴锥孔; 

将刀柄的键槽对准主轴端面键垂直伸入到主轴内,不可倾斜; 

右手按下换刀按钮,压缩空气从主轴内吹出以清洁主轴和刀柄,按住此按钮,直到刀柄锥面与主轴锥孔完全贴合后,松开按钮,刀柄即被自动夹紧,确认夹紧后方可松手; 

用手转动主轴检查刀柄是否正确装夹; 

卸刀柄时,先用左手握住刀柄,再用右手按换刀按钮,取下刀柄。 

\subsubsection{主轴的转速的调节} 

	用主轴倍率调节

	用S指令调节

	用带轮、变速箱、电机档位来调节



\begin{table}
		\caption{主轴转速的调节}
		\centering
	\begin{tabular}{|c|c|c|c|c|c|c|c|c|}
		\hline 
		&  \multicolumn{4}{c|}{1}& \multicolumn{4}{c|}{1} \\ 
		\hline 
		& 1 & 2 & 3 & 4 & 1 & 2 & 3 & 4 \\ 
		\hline 
		A&  &  &  &  &  &  &  &  \\ 
		\hline 
		B&  &  &  &  &  &  &  &  \\ 
		\hline 
	\end{tabular} 
\end{table}

\subsubsection{工件坐标系的设定及对刀}
设定工件坐标系的方法

用G54/G59来设置工件坐标系;

用G92来设置工件坐标系;

用G52来设置局部工件坐标系。

极坐标系及坐标系的旋转

常用对刀工具:

寻边器

分中棒

Z轴对刀仪

塞尺、量块

基准棒

“取中法”对刀:

X、Y向对刀 

A.	寻边器测头靠近工件的左侧; 


B.	用手轮使测头接触到工件左侧,记下机床坐标系中的X1坐标值; 

C.	抬起寻边器,让测头靠近工件右侧; 

D.	全手轮使测头接触到工件左侧,记下机床坐标系中的X2坐标值; 

E.	计算其设定值 X=(X1+X2)/2 

F.	同理可测得Y的设定值。

Z向对刀 

A.	安装加工刀具

B.	将Z轴设定器(或固定高度的对刀块)放置在工件上; 

C.	移动刀具,使刀具端面靠近 Z 轴设定器上表面; 

D.	用手轮使刀具端面接触到 Z 轴设定器上表面,直到其指针指示到零位; 

E.	记下机床坐标系中的Z1值,

F.	计算设定值Z=Z1-Z轴对刀仪高度      (坐标系设定在上表面上)

将测得的 X 、 Y 、 Z 值输入到机床G54-G59的参数中。

效验工件坐标系:

A.	在MDI方式下,执行: G54G90G1X0Y0F2000;

B.	检查刀具是不是在工件坐标系的正上方

C.	再执行:Z10.0F200;

D.	检查刀具是不是在其正上方10mm

注意:检查Z轴时,进给倍率先设为0,再用手调节并控制。

	用机内测量法简化“取中法”的计算


\subsection{实习内容及过程}

\subsubsection{集合、组织实习}
1、清查学生人数

2、文明安全生产讲解

3、实习内容说明
\subsubsection{开机15分钟}
1、由组长记录机床相关问题

2、开机前检查仔细

3、空转几分钟预热
\subsubsection{机床操作及编程}
1、教师演示基本操作

2、组长安排2人员操作机床(1人操作,1个指导)

3、其他人员自选图形编程

4、每人操作时间不得超过2小时

5、教师巡回指导
\subsubsection{操作点评及工件检测}
1、学生操作感想说明及自评

2、教师提问及点评

3、学生对工件自测

4、教师检测及评分
\subsubsection{准备下课}
1、清洁数控机床

2、正常关机

3、集合教师点评

\subsection{练习题及作业}
\begin{compactenum}[1、]
	\item 小结;
	\item 基本指令;
	\item 相关知识
	\item 机床操作
	\item 编程思路。
\end{compactenum}

\vfill
\subsection{加工准备与加工要求}
\subsubsection{加工准备}
\begin{enumerate}[1、]
\item 设备:数控铣床、加工中心。
\item 材料:45圆钢($\varnothing$110*35)。
\item 工具:活动扳手,平行垫铁,百分表,其它常用辅具。
\item 量具:外径千分尺(0~25、100~125,0.01),深度千分尺(0~25,0.01),R规。
\item 刀具:$\varnothing$10、$\varnothing$16、$\varnothing$14立铣刀、$\varnothing$64面铣刀。
\item 夹具:三爪自定心卡盘、螺杆压板、平口钳。
\end{enumerate}
\subsubsection{课题评分表}

{\noindent
%\begin{figure}[!hbtp]
%	\centering	
\footnotesize
\hspace{-3.9ex} \renewcommand\arraystretch{1.9}
\begin{tabu} to 0.5\textwidth {|cc|c|c|c|c|c|c|}
	\hline 
\multicolumn{2}{|c|}{工件编号}  &\multicolumn{2}{c}{} & \multicolumn{2}{|c}{总得分}   & \multicolumn{2}{|c|}{ }   \\ 
	\hline 
 \multicolumn{2}{|c|}{项目与配分} &\parbox{2ex}{序号}  & 技术要求 & 配分 & 评分标准 &  \parbox{4ex}{检测记录}& 得分 \\ 
	\hline 
	
 \multicolumn{2}{|c|}{\multirow{3}{*}{\parbox{10ex}{  找正
 		(25\%)} } }&1  &百分表的使用  &7 & 出错一处扣2分 &  &  \\ 
	\cline{3-8} 
&&2&虎钳及工件的找正  &9  &操作不正确全扣  &&  \\ \cline{3-8} 
&&3&工件上表面的找正  &9  &操作不正确全扣  &&  \\
	\hline 
	
 \multicolumn{2}{|c|}{\multirow{2}{*}{\parbox{10ex}{刀具的安装
 		(15\%)}
 } } &4 &刀具的安装  & 7 & 出错一次扣2分 &  &  \\ 
\cline{3-8} 
&&5&刀库中刀具的安装  &8  &出错一次扣2分&&  \\ 
\hline 	

 \multicolumn{2}{|c|}{\multirow{2}{*}{\parbox{10ex}{主轴的调速
			(15\%)}
} } &6 &主轴的转向  & 7 & 出错一次扣2分 &  &  \\ 
\cline{3-8} 
&&7&主轴转速的调节  &8  &出错一次扣2分&&  \\ 
\hline 

 \multicolumn{2}{|c|}{\multirow{3}{*}{\parbox{10ex}{  对刀
			(25\%)} } }&8  &XY向“取中法”对刀  &18 & 操作不正确全扣 &  &  \\ 
\cline{3-8} 
&&9&Z向对刀  &10  &操作不正确全扣  &&  \\ \cline{3-8} 
&&10&寻边器、Z轴对刀仪的使用  &7 &出错一处扣2分  &&  \\
\hline 

 \multicolumn{2}{|c|}{\multirow{2}{*}{\parbox{10ex}{机床操作
			(10\%)}
} } &11 &机床操作规范  & 5 & 出错一次扣2分 &  &  \\ 
\cline{3-8} 
&&12&工件刀具装夹  &5  &出错一次扣2分&&  \\ 
\hline 

 \multicolumn{2}{|c|}{\multirow{2}{*}{\parbox{10ex}{安全文明生产
 		(倒扣分)}
 } } &13  &安全操作  & 倒扣 & \multirow{2}{*}{\parbox{14ex}{安全事故停止操作或酌情扣分}}&  &  \\ 
\cline{3-5} \cline{7-8} 
&&14&机床整理  &倒扣  &  &  &\\ 
\hline 	
\end{tabu} }
%\end{figure}
\vfill