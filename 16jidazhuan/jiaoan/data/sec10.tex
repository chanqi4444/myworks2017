\jxhj{%教学后记
	}
\skrq{%授课日期
	2017年10月17日 4-5节}
\ktmq{%课题名称
	 子程序概述及Z向分层}
\jxmb{%教学目标,每行前面要加 \item
	\item 掌握子程序的概念;
	\item 掌握子程序命令的使用;
	\item 掌握使用子程序进行Z向分层的思路;
	\item 掌握掌握子程序的编程。 }
\jxzd{%教学重点,每行前面要加 \item
	\item Z向分层的思路;
	\item 子程序命令的使用。 }
\jxnd{%教学难点,每行前面要加 \item
	\item Z向分层的思路。 }
\jjff{%教学方法
	通过讲述、举例、演示法来说明;}

\makeshouye %制作教案首页

%%%%教学内容
\subsection{组织教学}
\begin{enumerate}[\hspace{2em}1、]
	\item 集中学生注意力;
	\item 清查学生人数;
	\item 维持课堂纪律;
\end{enumerate}
\subsection{复习导入及主要内容}
\begin{enumerate}[1、]
	\item 用刀补去残料的思路。
\item 去材料刀补的计算;
\item 多个刀补的编程;
\item 巩固粗/精加工刀补值的确定。
\end{enumerate}

\subsection{教学内容及过程}

\subsubsection{Fanuc上子程序的格式及调用}
[ 格式 ]
■子程序构成



■	子程序呼叫

[ 说明 ]
当主程序呼叫子程序时,它是一重子程序呼叫。因此,子程序可以做四重呼叫,如下图所示。                                                 

一个单个呼叫指令可以重复呼叫子程序最多到9999次。
对于兼容的编程装置,在第一个单节里,Nxxxx可以代替子程序O(或:)跟着的数字。在N后面的顺序号被认为是子程序号。

[ 注意 ]
1.	M98和M99信号不输出到机床。
2.	不到位址指定的子程序号,输出报警(No. 078)。

[ 举例 ]
☆M98 P51002;
这条指令指定“呼叫子程序(程序号1002)5次”。子程序呼叫指令(M98P\_\_\_)可以在移动指令单节中指定。
☆X1000.0 M98 P1200;
这个例子在X轴移动之后呼叫子程序(子程序号 1200)。
☆从主程序呼叫子程序的执行顺序
主程序                   子程序
N0010 O;               N0010 O;         
N0020 O;               N0020 O;         
N0030 M98 P21010;      N0030 M98 P21010;                 
N0040 O;               N0040 O;             
N0050 O;               N0050 O;
子程序可以象主程序呼叫子程序一样呼叫另一个子程序。

[ 特殊用途 ] 
■主程序中使用M99
如果在主程序中执行M99,控制返回主程序开头。举例说,/M99放在程序中并执行M99;在主程序的适当位置设定选择性单节跳跃功能,在执行主程序时关掉。当执行M99时,控制返回到主程序的开头,然后主程序从头开始重复执行。
当选择性单节跳跃功能设定关时,重复执行程序。当选择性单节跳跃功能设定开时,/M99单节被跳过;控制进入下一个单节继续执行。
二、使用子程序的场合:
1、同一零件上有重复加工的部位
2、批量加工,一次加工多个相同的工件
3、同一个零件上有多个不同形状的加工轮廓
可使程序看起来方便,简单
不使用的场合:
自动编程不用子程序:
1、多程序传输不方便
2、自动编程子程序完成后,修改的内容多,
三、编程实例
方形:
O0001
G54G17G40G49G90
M3S500
G1Z30.F2000
X-50.Y0
Z5.0
Z0F200
D1M98P40002
G1Z30.F2000
M4
M30
O0002
G91G1Z-5.0
G90G41X-40.Y0
G3X-30.Y0R10.
G1Y30.
….
M99

\subsection{课堂小结}
\begin{enumerate}[1、]
	\item 加工轮廓的处理;
	\item 极坐标;
	\item 加工工序。
\end{enumerate}

\vfill
\subsection{布置作业}
\begin{enumerate}[1、]
	\item 自选一零件图, 写出其工艺与程序。 
\end{enumerate}
\vfill