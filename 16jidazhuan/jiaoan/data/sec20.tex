\jxhj{%教学后记
	}
\skrq{%授课日期
	2017年11月21日 4-5节}
\ktmq{%课题名称
	 孔加工概述}
\jxmb{%教学目标,每行前面要加 \item
	\item 掌握孔加工的加工工艺;
	\item 掌握孔加工的六个动作;
	\item 了解用子程序进行孔加工的编程;
	\item 掌握G90/G91在孔加工中的区别。}
\jxzd{%教学重点,每行前面要加 \item
	\item 孔加工的六个动作;
	\item G90/G91在孔加工中的区别。 }
\jxnd{%教学难点,每行前面要加 \item
	\item 用子程序进行孔加工的编程。 }
\jjff{%教学方法
	通过讲述、举例、演示法来说明;}

\makeshouye %制作教案首页

%%%%教学内容
\subsection{组织教学}
\begin{enumerate}[\hspace{2em}1、]
	\item 集中学生注意力;
	\item 清查学生人数;
	\item 维持课堂纪律;
\end{enumerate}

\subsection{复习导入及主要内容}
\begin{enumerate}[1、]
\item 岛屿型腔加工的下刀方式;
\item 岛屿槽去残料的方式;
\item 用G91螺线下刀及Z向分层;
\item 编写岛屿型腔的程序。
\end{enumerate}

\subsection{教学内容及过程}
\subsubsection{孔加工工艺}
1、钻孔

刀具:中心钻(φ3、φ5)

结构形状

钻削深度:

2、钻孔

刀具:直柄麻花钻、锥柄麻花钻、群钻(各种直径)
钻削深度:超出量,0.3倍刀具直径。

3、扩孔

刀具:

4、镗孔:镗刀

5、铰孔加工

刀具:机用绞刀

超出量:刀具值 一般2mm

6、攻丝:

柔性攻丝、刚性攻丝




\subsubsection{孔加工的三个平面及六个动作}
1、三个平面:

初始平面

R点平面

孔底平面

2、六个动作:

XY向快速定位

Z向快速下刀到R点平面

Z向慢速下刀切削

孔底动作

返回到R点平面

快速返回的初始平面

\subsubsection{加工实例}

在数控机床上加工如图所示的零件,是完成孔加工的程序编写。共四个孔,横向间距为80mm,纵向间距为60mm。孔的直径为10H7,有效深度为20mm。

1、刀具的选择:

\diameter 3的中心钻、\diameter9.8的直柄麻花钻、\diameter10H7的机用绞刀。

2、加工工序

3、加工程序


\begin{lstlisting}
02
G54G17G40G90
M3S1200
G1Z30.F2000
X-40.Y30.
M98P21
X40
M98P21
Y-30
M98P21
X-40
M98P21
G1Z30.F2000
M5
M99
\end{lstlisting}
\begin{lstlisting}
O21(G90方式)
G1Z5.0F2000
Z-6.0F80
Z5.0
Z30.F2000
M99
\end{lstlisting}
\begin{lstlisting}
O21(G91方式)
G91G1Z-25.0F2000
Z-11.F80
Z11.
Z25.F2000
G90
M99
\end{lstlisting}

\begin{lstlisting}
O3
G55G17G40G90
M3S500
G1Z30.F2000
X-40.Y30.
M98P31
X40
M98P31
Y-30
M98P31
X-40
M98P31
G1Z30.F2000
M5
M99
\end{lstlisting}

\begin{lstlisting}
O4
G56G17G40G90
M3S500
G1Z30.F2000
X-40.Y30.
M98P41
X40
M98P41
Y-30
M98P41
X-40
M98P41
G1Z30.F2000
M5
M99
……
\end{lstlisting}

\subsubsection{孔加工固定循环}

G99/G98 G90/G91 G81-G89 X Y Z R Q P L


\subsection{课堂小结}
\begin{enumerate}[1、]
\item 孔加工刀具;
\item 孔加工方式;
\item 铣孔与钻孔;
\item 孔加工编程。
\end{enumerate}

\vfill
\subsection{布置作业}
\begin{enumerate}[1、]
	\item 编写上面的程序。
\end{enumerate}
\vfill