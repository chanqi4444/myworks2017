\jxhj{%教学后记
}
\skrq{%授课日期
2017~~|~~9.5~~|~~1-3节}
\ktmq{%课题名称
安全操作及机床面板认识}
\jxmb{%教学目标,每行前面要加\item 
\item 明确数控铣/加工中心的文明生产及安全操作规程;
\item 掌握数控铣床/加工中心的组成及坐标系的判定;
\item 明确数控铣床/加工中心MDI面板按键的作用;
\item 掌握回零操作、轴移动操作及开/关机的步骤。}
\jxzd{%教学重点,每行前面要加\item 
\item 明确数控铣床/加工中心MDI面板按键的作用;
\item 掌握回零操作、轴移动操作及开/关机的步骤。}
\jxnd{%教学难点,每行前面要加\item 
\item 掌握回零操作、轴移动操作及开/关机的步骤;}
\jjff{%教学方法
通过讲述、举例、演示法来说明;}

\makeshouye%制作教案首页

%%%%教学内容
\subsection{实习教学要求}
\begin{compactenum}[1、]
\item 明确数控铣/加工中心的文明生产及安全操作规程;
\item 掌握数控铣床/加工中心的组成及坐标系的判定;
\item 明确数控铣床/加工中心MDI面板按键的作用;
\item 掌握回零操作、轴移动操作及开/关机的步骤。
\end{compactenum}

\subsection{相关工艺}
\subsubsection{文明安全生产要求}
\begin{compactenum}[1、]
\item 精神饱满、文明交流;
\item 统一工作服;
\item 操作台只站一个人、在规定的区域里活动;
\item 工件、量具等摆放整齐有序;
\item 精密量具放在盒子里;
\item 爱护机床卫生、保持车间整洁;
\item 严格按机床安全操作规程操作;
\item 禁止修改系统参数;
\item 实行“一人一机上机操作”;
\item 穿合适的工作服,禁止戴手套、穿拖鞋;
\item 女生盘好头发;
\item 加工中禁止离机。
\end{compactenum}


\subsubsection{安全操作规程}\marginpar{具体后面讲解}
\begin{compactenum}[1、]
\item 开机;
\item 程序调试;
\item 加工中;
\item 关机。
\end{compactenum}

\subsubsection{数控铣/加工中心的组成}
\begin{compactenum}[1、]
\item 主轴箱主轴;
\item 控制面板;
\item 电气柜;
\item 立柱床身;
\item 工作台;
\item 冷却液箱;
\item 刀库。
\end{compactenum}

\subsubsection{机床面板及数控系统界面}
\begin{compactenum}[1、]
\item 加工方式:手动、MDI、自动、编辑、回零、DNC等;
\item 进给倍率、快速倍率、主轴倍率;
\item 复位、进给保持、循环启动;
\item 轴移动、主轴正转/反转/停止、切削液开/关、刀库正/反转;
\item 跳段、单段、选择停、空运行、机床锁住、Z轴锁住、M功能锁住;
\item 急停、手轮;
\item 地址键:OPGR……;
\item 数据键:1234……;
\item 功能键:POS、PROG、OFFSETSETING、SYSTEM、MESSAGE、GRAPH;
\item 编辑键:SHIFT、CAN、INPUT、ALTER、INSERT、DELETE、EOB;
\item 坐标显示:绝对、相对、总和等;
\item 程序编辑与管理:程序显示、程序信息、背景编程;
\item 加工参数设定:半径、长度、工件坐标系;
\item 图形模拟;
\item 帮助及报警。
\end{compactenum}

\subsubsection{机床基本操作}
\begin{compactenum}[1、]
\item 开机:开机前检查——外部电源——机床电源——取消急停——复位;
\item 回零:回零方式——调节快速倍率——Z+——X+——Y+——各轴指示灯亮;
\item 手动移动:手动方式——调节进给倍率——X/Y/Z轴;
\item 手轮移动:手轮方式——选择轴——选择倍率——手摇手轮;
\item 快速移动:快速方式——调节快速倍率——X/Y/Z轴。
\end{compactenum}
\textbf{注意:}
\begin{compactenum}[1、]
\item 开机中禁止按任何按键;
\item 开机后确认显示正常、无报警、风扇电机转动正常;
\item 禁止在零点附近回零;
\item 转动手轮不能过快,以不超过5r/s为宜;
\item 手轮倍率应以X100、X10、X1的顺序操作;
\item 移动轴时应先确认好刀具的移动方向;
\item 超程时解除超程。
\end{compactenum}

\subsection{实习内容及过程}

\subsubsection{集合、组织实习}
1、清查学生人数

2、文明安全生产讲解

3、实习内容说明
\subsubsection{开机15分钟}
1、由组长记录机床相关问题

2、开机前检查仔细

3、空转几分钟预热
\subsubsection{机床操作及编程}
1、教师演示基本操作

2、组长安排2人员操作机床(1人操作,1个指导)

3、其他人员自选图形编程

4、每人操作时间不得超过2小时

5、教师巡回指导
\subsubsection{操作点评及工件检测}
1、学生操作感想说明及自评

2、教师提问及点评

3、学生对工件自测

4、教师检测及评分
\subsubsection{准备下课}
1、清洁数控机床

2、正常关机

3、集合教师点评

\subsection{练习题及作业}
\begin{compactenum}[1、]
\item 写出你所操作的机床的主要技术参数。
\item 按X+、Y+工作台向什么方向称动,与坐标系有什么关系,为什么?
\item 数控铣床开机之后为什么要执行回机床参考点的操作?如何操作?
\item 在启动数控铣床前,操作者要做哪些检查?
\item 什么叫“超程”?如何解除超程报警?
\item 在数控铣床运行过程中,当出现异常情况时如何处理
\end{compactenum}

\vfill
\subsection{加工准备与加工要求}
\subsubsection{加工准备}
\begin{enumerate}[1、]
\item 设备:数控铣床、加工中心。
\item 材料:45圆钢(Ф82*50)。
\item  工具:活动扳手,平行垫铁,百分表,其它常用辅具。
\item  
量具:外径千分尺(0~25、100~125,0.01),深度千分尺(0~25,0.01),R规。
\item  刀具:Ф10、Ф16、Ф14立铣刀、Ф64面铣刀。
\item  夹具:三爪自定心卡盘、螺杆压板、平口钳。
\end{enumerate}
\subsubsection{课题评分表}

{\noindent
\begin{figure}[!hbtp]
%\centering
\footnotesize
\hspace{-3.4ex}\renewcommand\arraystretch{1.9}
\begin{tabu}to0.45\textwidth{|cc|c|c|c|c|c|c|}
\hline
\multicolumn{2}{|c|}{工件编号}&\multicolumn{2}{c}{}&
\multicolumn{2}{|c}{总得分}&\multicolumn{2}{|c|}{}\\
\hline
\multicolumn{2}{|c|}{项目与配分}&\parbox{2ex}{序号}&技术要求&配分
&评分标准&\parbox{4ex}{检测记录}&得分\\
\hline
\multicolumn{2}{|c|}{\multirow{2}{*}{\parbox{10ex}{文明生产
			(20\%)}}}&1&工作服&8&未穿禁止进车间并全扣&&\\
\cline{3-8}
&&2&工具、量具摆放整齐&8&不整齐有序全扣&&\\ \cline{3-8}
&&3&其它&4&不守纪律全扣&&\\
\hline

\multicolumn{2}{|c|}{\multirow{1}{*}{\parbox{10ex}{安全操作规程
			(20\%)}}}&4&操作安全&20&酌情扣分&&\\[.3cm] \hline

\multicolumn{2}{|c|}{\multirow{1}{*}{\parbox{10ex}{加工中心组成
			(10\%)}}}&5&说出各部分名称&10&出错一处扣2分&&\\[.3cm]  \hline

\multicolumn{2}{|c|}{\multirow{2}{*}{\parbox{10ex}{面板系统界面
			(20\%)}}}&6&操作面板&10&出错一处扣2分&&\\ \cline{3-8}
&&7&系统界面的认识	&10&出错一处扣2分&&\\ \hline

\multicolumn{2}{|c|}{\multirow{4}{*}{\parbox{10ex}{机床操作
			(30\%)}}}&8&手动、手轮、快速&10&出错一处扣2分&&\\ \cline{3-8}
&&9&点的定位&10&出错一处扣2分&&\\ \cline{3-8}		
&&10&机床操作规范&5&出错一处扣2分&&\\ \cline{3-8}		
&&11&工件刀具装夹	&5&出错一处扣2分&&\\ \hline	

\multicolumn{2}{|c|}{\multirow{2}{*}{\parbox{10ex}{安全文明生产
(倒扣分)}
}}&12&安全操作&倒扣&
\multirow{2}{*}{\parbox{14ex}{安全事故停止操作或酌情扣分}}&&\\
\cline{3-5}\cline{7-8}
&&13&机床整理&倒扣&&&\\
\hline
\end{tabu}
\end{figure}}