\jxhj{%教学后记
	}
\skrq{%授课日期
	2017年 | 5月9日 |5月16日|5月23日|5月30日|6月7日| 1-3节}
\ktmq{%课题名称
	 面铣及手动铣削}
\jxmb{%教学目标,每行前面要加 \item
	\item 掌握平面加工工艺;
	\item 掌握平加工程序;
	\item 掌握手动铣削;
    \item 掌握“取中法”的对刀过程。}
\jxzd{%教学重点,每行前面要加 \item
	\item 掌握平加工程序;
	\item 掌握平面加工工艺。}
\jxnd{%教学难点,每行前面要加 \item
	\item 掌握平面加工工艺。}
\jjff{%教学方法
	通过讲述、举例、演示法来说明;}

\makeshouye %制作教案首页

%%%%教学内容
\subsection{实习教学要求}
\begin{enumerate}[1、]
\item 掌握平面加工工艺;
\item 掌握平加工程序;
\item 掌握手动铣削;
\item 掌握“取中法”的对刀过程。
\end{enumerate}


\subsection{相关工艺知识与编程}

\subsubsection{相关键和按钮的功能}
单段:按下后,指示灯亮,在自动加工时按一次循环启动执行一条程序段。

跳步:打开后,机床不执行带“/”的程序段。

空运行:打开此功能后,机床以最快的速度移动,用于校验程序。

机床锁住:开开此功能后,机床仅进行脉冲分配,但不输出脉冲到伺服电机上,即位置显示与程序同步,但机床不移动,M、S、T代码执行。

选择停:打开此功能后,在自动运行时,遇到M01,程序晢停,冷却关断,按“循环启动”继续执行下段程序。

辅助功能锁住:在自动运行时,M、S、T指令不执行。

循环启动:启动加工程序。

进给保持:在自动运行状态下,停止进给,即程序晢停,按循环启动程序接着执行。

Siemens、Fanuc、KND都有此功能按键,位置不同。

\subsubsection{自动运行程序}
1、先对刀,输入好程序

2、选择“自动”方式

3、从存储的程序中选择程序

A、按下[PROG]键,显示程序

B、按下地址[O]键

C、输入程序号

D、按下方向键(向下的)

4、设定好各功能按扭、进给速度等

5、按下[循环启动],程序运行,指示灯亮,程序运行结束,指示灯灭。

6、中途停止或者取消自动运行:

A、用[进给保持],程序晢停

B、按键盘上的复位键,程序结束,机床停止

C、按急停键


\subsubsection{铣平面}

几乎所的有零件都要进行平面铣削,故平面铣削是加工的一项主要内容。

1、采用刀具,一般用面铣刀,其直径为40-250mm螺旋角为10度,刀齿数为10-20个。小平面也可采用大直径的立铣刀。

2、切削深度,铣平面要保证去掉毛坯上的凸凹不平,如要提高表面质量,平面可再铣0.2mm。

3、走刀路径:直线单方向,直线往返,走圆形。
切削宽度:刀具直径的60%-80%

4、工件坐标系原点:

毛坯上表面:(后面的程序Z方向不好计算)

加工后上表面:(下刀到Z0)

5、参考程序:

\begin{lstlisting}
Fanuc:
O0001;
G54G17G40 G90G80;
M03 S300;
G0 X-60. Y-60.0;
Z10.0;
G01 Z0 F100.0;
N10 G01 G91 X120.0 F300.0;
Y10.0;
X-120.0;
Y10.0;
#1=#1+1;
IF [#1 < 6 ] GOTO10;(中间没有空隔)
G90 Z10.0;
G00 Z50.0;
X0 Y0;
M05 
M30
\end{lstlisting}
圆形路径自己写。


\subsubsection{编写程序的基本思路}
程序初始化(安全保护)------辅助准备(换刀、主轴启动、切削液开等) ------提刀到安全平面------定位到下刀点------快速下刀------工进下刀------加工轮廓------提刀-------快速提刀到安全平面-----程序结束(换刀、主轴停止、切削液关、程序返回等)


\subsection{实习内容及过程}

\subsubsection{集合、组织实习}
1、清查学生人数

2、文明安全生产讲解

3、实习内容说明
\subsubsection{开机15分钟}
1、由组长记录机床相关问题

2、开机前检查仔细

3、空转几分钟预热
\subsubsection{机床操作及编程}
1、教师演示基本操作

2、组长安排2人员操作机床(1人操作,1个指导)

3、其他人员自选图形编程

4、每人操作时间不得超过2小时

5、教师巡回指导
\subsubsection{操作点评及工件检测}
1、学生操作感想说明及自评

2、教师提问及点评

3、学生对工件自测

4、教师检测及评分
\subsubsection{准备下课}
1、清洁数控机床

2、正常关机

3、集合教师点评


\vfill
\subsection{加工准备与加工要求}
\subsubsection{加工准备}
\begin{enumerate}[1、]
	\item 设备:数控铣床、加工中心。
	\item 材料:45圆钢($\varnothing$110*35)。
	\item 工具:活动扳手,平行垫铁,百分表,其它常用辅具。
	\item 量具:外径千分尺(0~25、100~125,0.01),深度千分尺(0~25,0.01),R规。
	\item 刀具:$\varnothing$10、$\varnothing$16、$\varnothing$14立铣刀、$\varnothing$64面铣刀。
	\item 夹具:三爪自定心卡盘、螺杆压板、平口钳。
\end{enumerate}
\subsubsection{课题评分表}

{\noindent
%\begin{figure}[!hbtp]
%	\centering	
\footnotesize
\hspace{-2.8ex} \renewcommand\arraystretch{1.9}
\begin{tabu} to 0.5\textwidth {|cc|c|c|c|c|c|c|}
	\hline 
	\multicolumn{2}{|c|}{工件编号}  &\multicolumn{2}{c}{} & \multicolumn{2}{|c}{总得分}   & \multicolumn{2}{|c|}{ }   \\ 
	\hline 
	\multicolumn{2}{|c|}{项目与配分} &\parbox{2ex}{序号}  & 技术要求 & 配分 & 评分标准 &  \parbox{4ex}{检测记录}& 得分 \\ 
	\hline 
	\multirow{4}{*}{ \parbox{4ex}{工件加工 (80)}} &\multicolumn{1}{|c|}{上面}  & 1 &面铣  & 4 & 超差全扣 & & \\ 
	\cline{2-8}  
	&\multicolumn{1}{|c|}{上面}   & 2 &尺寸1  & 12 & 超差全扣 & & \\ 
	\cline{2-8} 
	&\multicolumn{1}{|c|}{上面}  & 3 &尺寸2  & 12& 超差全扣 & & \\ 
	\cline{2-8} 
	&\multicolumn{1}{|c|}{上面}   & 4&椭圆  & 30 & 超差全扣 & & \\ 
	\hline 
	\multicolumn{2}{|c|}{\multirow{2}{*}{\parbox{10ex}{程序与工艺
				(10\%)} } }&5  &程序正确合理  & 5 & 每错一处扣2分 &  &  \\ 
	\cline{3-8} 
	&&6&加工工序卡  &5  &不合理每处扣2分  &&  \\ 
	\hline 
	\multicolumn{2}{|c|}{\multirow{2}{*}{\parbox{10ex}{机床操作
				(10\%)}
	} } &7 &机床操作规范  & 5 & 出错一次扣2分 &  &  \\ 
	\cline{3-8} 
	&&8&工件刀具装夹  &5  &出错一次扣2分&&  \\ 
	\hline 	
	\multicolumn{2}{|c|}{\multirow{2}{*}{\parbox{10ex}{安全文明生产
				(倒扣分)}
	} } &9  &安全操作  & 倒扣 & \multirow{2}{*}{\parbox{14ex}{安全事故停止操作或酌情扣分}}&  &  \\ 
	\cline{3-5} \cline{7-8} 
	&&10&机床整理  &倒扣  &  &  &\\ 
	\hline 	
\end{tabu} }
%\end{figure}
\vfill