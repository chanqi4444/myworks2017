\jxhj{%教学后记
	 }
\skrq{%授课日期
	2017年~~9月5日~~4-5节   }
\ktmq{%课题名称
	数控编程概术 }
\jxmb{%教学目标,每行前面要加 \item
	\item 了解数控技术的基本知识;
	\item 掌握数控机床的分类;
	\item 了解制造自动化技术的发展;}
\jxzd{%教学重点,每行前面要加 \item
	\item 了解数控技术的基本知识;
	\item 数控机床的分类;}
\jxnd{%教学难点,每行前面要加 \item
	\item 数控技术的基本知识;}
\jjff{%教学方法
	通过讲述、举例、演示法来说明;}

\makeshouye %制作教案首页

%%%%教学内容
\subsection{组织教学}
\begin{enumerate}[\hspace{2em}1、]
	\setlength{\itemsep}{0pt}
	\item 集中学生注意力;
	\item 清查学生人数;
	\item 维持课堂纪律;
\end{enumerate}
\subsection{复习导入及主要内容}
\begin{enumerate}[\hspace{2em}1、]
\item 自我介绍
\item 企业需要什么样的数控人才:\\
A:人品好:有道德、有理想、乐于助人。\\
B:精神面貌好:胆大心细,勇于思考、
能吃苦,不怕累、不怕脏。\\
C:能干事,有水平、速度快。\\
D:积极上进,乐于创新
\item 学习目标及要求:\\
A:会操作数控机床。快,准、细心。\\
B:能进行简单零件的工艺处理。会、经验积累。 \\
C:能进行简单零件的编程。思路清晰、编程细心、改错准快。\\
\end{enumerate}
跟着老师、多练习(作业)、准备充分、多想多问多看。
\subsection{教学内容及过程}
\subsubsection{制造自动化技术的发展} \marginpar{说明介绍}
1、人工控制——手动操作

2、刚性自动化——在20世纪40~50年代已相当成熟。应用传统的机械设计与制造工艺方法,采用专用机床和组合机床、自动单机或自动化生产线进行大批量生产。其特征是高生产率和刚性结构,很难实现生产产品的改变。引入的新技术包括继电器程序控制、组合机床等。

3、数控加工——包括数控(NC)和计算机数控(CNC)。特点是柔性好、加工质量高,适应于多品种、中小批量(包括单件产品)的生产。引入的新技术包括数控技术、计算机编程技术等。

4、柔性制造——其特征强调制造过程的柔性和高效率,适应于多品种、中小批量的生产。涉及的技术包括成组技术(GT)、计算机直接数控和分布式数控(DNC)、柔性制造单元(FMC)、柔性制造系统(FMS)、柔性加工线(FML)、离散系统理论和方法、仿真技术、车间计划与控制、制造过程监控技术、计算机控制与通信网络等。

5、计算机集成制造系统(CIMS)——其特征是强调全过程的系统性和集成性,以解决现代企业生存与竞争的 TQCS (Time-提供产品的时间,Quality-产品的质量,Cost-产品的成本,Service-产品的服务)问题。CIMS涉及的学科非常广泛,包括现代制造技术、管理技术、计算机技术、信息技术、自动化技术和系统工程等。

6、新的制造自动化模式——如智能制造、敏捷制造、虚拟制造、网络制造、全球制造、绿色制造等。

\subsubsection{数控技术的基本概念} \marginpar{互动提问}
1、数控技术

数控(Numerical Control)技术是用数字化的信息对某一对象进行控制的技术。
控制对象可以是位移、角度、速度等机械量,也可以是温度、压力流量、颜色等物理量,这些量的大小不仅是可以测量的,而且可以经A/D或D/A转换,用数字信号来表示。数控技术是近代发展起来的一种自动控制技术,是机械加工现代化的重要基础与关键技术。

2、数控加工

数控加工是指采用数字信息对零件加工过程进行定义,并控制机床进行自动运行的一种自动化加工方法。

数控加工是20世纪40年代后期为适应加工复杂外形零件而发展起来的一种自动化技术。1947年,美国帕森斯(Parsons)公司为了精确地制作直升机机翼、浆叶和飞机框架,提出了用数字信息来控制机床自动加工外复杂零件的设想。1949年美国空军为了能在短时间内制造出经常变更设计的火箭零件,与帕森斯公司和麻省理工学院(MIT)伺服机构研究所合作,于1952年研制成功世界上第一台数控机床——三坐标立式铣床,可控制铣刀进行连续空间曲面的加工,揭开了,数控加工技术的序幕。

3、数控机床

数控机床就是采用了数控技术的机床。
数控机床将零件加工过程所需的各种操作和步骤以及刀具与工件之间的相对位移量都用数字化的代码来表示,由编程人员编制成规定的加工程序,通过输入介质(磁盘等)送入计算机控制系统,由计算机对输入的信息进行处理与运算,发出各种指令来控制机床的运动,使机床自动地加工出所需要的零件。

特点有:高精度、高柔性、高效率、减轻工人的劳动强度,改善了劳动条件,有良好的经济效益,有利于生产的管理和现代化。

4、数控编程

数控编程(NC Programming)就是生成用数控机床进行零件加工数控程序的过程。
数控程序  是由一系列程序段组成,把零件的加工过程、切削用量、位移数据以及各种辅助操作,按机床的操作和运动顺序,用机床规定的指令及程序格式排列而成的一个有序指令集。如 N10 G00 X200.0 Y-39.0 M03 。

\subsubsection{数控机床的组成与工作过程}
1、数控机床的组成

A、输入输出设备

实现编制程序、输入程序、输入数据以及显示、存储和打印等功能。(如键盘、CRT显示器等)

B、数控系统

是数控机床的“大脑”和“核心”,通常由一台通用或专用计算机构成。它的功能是接受输入装置输入的加工信息,经过数控系统中的系统软件或逻辑电路进行译码、运算和逻辑处理后,发出相应的各种信号和指令给伺服系统,通过伺服系统控制机床的各个运用部件按规定要求动作。

C、伺服系统

伺服系统接收来自数控系统的指令信息,严格按指令信息的要求驱动机床的运动部件动作,以加工出符合要求的零件。伺服系统的伺服精度和动态响应是影响数控机床的加工精度、表面质量和生产率的重要因素之一。

D、机床本体

机床本体是数控机床的主体,包括:床身、立柱等去支承部件;主轴等运动部件;工件台、刀架以及运动执行部件、传动部件;此外还有冷却、润滑、转位和夹紧等辅助装置。

2、数控机床的工作过程

A、准备阶段

根据加工零件的图纸,确定有关加工数据(刀具轨迹坐标点、加工的切削用量、刀具尺寸信息等)根据工艺方案、夹具选用、刀具类型选择等确定有关期货辅助信息。

B、编程阶段

C、准备信息载体

D、加工阶段

加工




\subsubsection{数控机床的种类}
1、按工艺用途分类

A、金属切削类数控机床    数控车床、数控铣床、数控磨床、数控镗床以及加工中心。

B、金属成型类数控机床    数控折弯机、数控组合冲床、数控弯管机、数控回转头压力机等。这类机床起步晚,但目前发展很快。

C、数控特种加工机床    数控线切割机床、数控电火花机床、数控火焰切割机床、数控激光切割机床等。

D、其它类型数机床    数控三坐标测量机等。

2、按运动方式分类

A、点位控制数控机床    刀具点到点

B、直线控制数控机床    刀具平行于某一坐标轴移动

C、轮廓控制数控机床    刀具沿两个或两个以上的坐标轴同时移动。

3、按控制方式分类

A、开环控制系统

B、半闭环控制系统

C、闭环数控系统

4、按数控系统功能水平分类

数控机床按数控系统的功能水平可分为低、中、高三档。不同时期其划分标准也不同:

分辨率和进给速度

伺服进给类型

联运轴数

通信能力

显示功能

内装PLC

主CPU

高速、高效、高精度、高可靠性

模块化、智能化、柔性化和集成化



\subsubsection{数控机床的坐标系}
\subsection{课堂小结}
\begin{enumerate}[1、]
\item 制造自动化的发展;
\item 数控技术的基本概念;
\item 数控机床的组成与工作过程;
\item 数控机床的种类;
\item 数控机床的坐标系。
\end{enumerate}
\subsection{布置作业}
\begin{enumerate}[1、]
\item 简述数控机床的组成部分?
\item 简述数控机床的分类?
\item 数控机床有哪些特点?
\end{enumerate}
\vfill
