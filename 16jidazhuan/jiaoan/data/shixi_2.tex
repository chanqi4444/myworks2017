\jxhj{%教学后记
}
\skrq{%授课日期
2017~~|~~9.11~~|~~1-3节}
\ktmq{%课题名称
程序手工录入、编辑及刀路模拟}
\jxmb{%教学目标,每行前面要加\item 
\item 掌握MDI键盘各按键的作用;
\item 掌握程序的录入与编辑;
\item 认识机床面版及系统界面;
\item 掌握程序的检查及模拟。
}
\jxzd{%教学重点,每行前面要加\item 
\item 掌握MDI键盘各按键的作用;
\item 掌握程序的录入与编辑。}
\jxnd{%教学难点,每行前面要加\item 
\item 掌握MDI键盘各按键的作用.}
\jjff{%教学方法
通过讲述、举例、演示法来说明;}

\makeshouye%制作教案首页

%%%%教学内容
\subsection{实习教学要求}
\begin{compactenum}[1、]
\item 掌握MDI键盘各按键的作用;
\item 掌握程序的录入与编辑;
\item 认识机床面版及系统界面;
\item 掌握程序的检查及模拟。
\end{compactenum}


\subsection{相关工艺}
\subsubsection{MDI键盘说明}
\begin{enumerate}[1、]
\item 地址键:O、N、G……
      \subitem  EOB——程序段结束符,显示为“;”。
\item 数据键:1、2、3……
\item 功能键:
\subitem POS——在CRT中显示坐标值。
\subitem PROG——CRT将进入程序编辑和显示界面
\subitem OFFSET SETTING——CRT将进入加工参数设定界面
\subitem SYSTEM——CRT将进入系统参数设定界面
\subitem MESSAGE——CRT将进入信息(如报警)界面
\subitem CUSTOM-GRAPH——图行模拟
\item 	编辑键:
\subitem SHIFT——输入字符切换
\subitem CAN——删除缓存中的字符
\subitem INPUT——输入机床数据
\subitem ALTER——字符替换
\subitem INSERT——插入字
\subitem DELETE——删除
\subitem HELP——获取帮助
\subitem RESET——系统复位
	\item 方向及换页键

\end{enumerate}


\subsubsection{程序录入与编辑}
\begin{enumerate}[1、]
\item 	程序管理
\subitem 检索程序:\\编辑方式——PROG界面——O+程序号——向下方向键

\subitem 	新建程序:\\编辑方式——PROG界面——O+程序号——INSRT——EOB
\subitem  删除程序:\\编辑方式——PROG界面——O+程序号——DELET——EXEC
\subitem 程序信息:\\编辑方式——PROG界面——[DIR]——[DIR+]
\item 	程序录入编辑
\subitem 字的检索:\\地址+数据——[SRH↓]    (或直接用方向键)
\subitem 字的插入:\\检索或定位到插入字之前——地址+数据——INSTRT	(可一次插入多个字)
\subitem 字的替换:\\检索或定位到插入字之前——地址+数据——ALTER
\subitem 字的删除:\\检索或定位到插入字之前——DELETE
\item 	背景编程
\subitem 进入背景编程:\\PROG——[OPRT]——[BG-EDT]——屏幕的左上显示[BG-EDIT]
\subitem 退出背景编程:\\PROG——[OPRT]——[BG-EDT]——屏幕的左上 [BG-EDIT] 不显示
\end{enumerate}

\subsubsection{机床面板按键}
\begin{enumerate}[1、]
\item 跳段:使注释符号“/”有效,即跳过“/”开头的程序段
\item 单段:加工时按循环启动执行一条程序段
\item 空运行:循环启动时以空运行速度运行,程序中的F无效
\item 机床锁定:锁定机床,即机床不动,系统模拟运行。
\item 选择停:使程序中“M01”有效
\item Z轴锁定:锁定Z轴不动

\end{enumerate}

\subsubsection{程序的图形模拟}
CUSTOM GRAPH——设定好显示参数——GRAPH——启动程序

注意:可结合上面的自动加工控制功能来模拟


\subsubsection{需要回零点的情况}
回零可重新建立机床坐标系
\begin{compactenum}[1、]
\item 开机后;
\item 机床断电后再次接通数控系统电源;
\item 紧急停止按钮按下后;
\item Z轴锁住后;
\item 机床锁住后;
\item 超程取消后。
\end{compactenum}

\subsection{实习内容及过程}

\subsubsection{集合、组织实习}
1、清查学生人数

2、文明安全生产讲解

3、实习内容说明
\subsubsection{开机15分钟}
1、由组长记录机床相关问题

2、开机前检查仔细

3、空转几分钟预热
\subsubsection{机床操作及编程}
1、教师演示基本操作

2、组长安排2人员操作机床(1人操作,1个指导)

3、其他人员自选图形编程

4、每人操作时间不得超过2小时

5、教师巡回指导
\subsubsection{操作点评及工件检测}
1、学生操作感想说明及自评

2、教师提问及点评

3、学生对工件自测

4、教师检测及评分
\subsubsection{准备下课}
1、清洁数控机床

2、正常关机

3、集合教师点评

\subsection{练习题及作业}
FANUC机床输入P8页的程序

SIEMENS机床输入P85页的程序

宏程序输入:

\begin{lstlisting}
O1000;
N10 #100=1.0;
N20 #101=0;
N30 #102=361.0;
N40 #103=45.0;
N50 #104=25.0;
N60 #105=-10.0;
N70 G54 G17 G49 G90 G40;
N80 M03;
N90 G00 Z50.0;
N100    X[#103+30.0] Y0;
N110    Z5.0;
N120 G01 Z#105 F100;
N130 G01 G42 X[#103+15.0] Y-15.0 D01 F200;
N140 G02 X#103 Y0 R15.0;
N150 #114=#101;
N160 WHILE [ #114 LT #102 ] DO1;
N170 #112=#103*COS[#114];
N180 #113=#104*COS[#114];
N190 G01 X[ROUND[#112]] Y[ROUND[#113]];
N200 #114=#114+#100;
N210 END1
N220 G02 X[#103+15.0] Y15.0 R15.0;
N230 G40 G01 X[#103+30.0] Y0;
N240 G00 Z50.0;
N250 M05;
N260 M30;

\end{lstlisting}
注意事项

1、不要乱删程序

2、注意安全


\vfill
\subsection{加工准备与加工要求}
\subsubsection{加工准备}
\begin{enumerate}[1、]
\item 设备:数控铣床、加工中心。
\item 材料:45圆钢(Ф82*50)。
\item  工具:活动扳手,平行垫铁,百分表,其它常用辅具。
\item  
量具:外径千分尺(0~25、100~125,0.01),深度千分尺(0~25,0.01),R规。
\item  刀具:Ф10、Ф16、Ф14立铣刀、Ф64面铣刀。
\item  夹具:三爪自定心卡盘、螺杆压板、平口钳。
\end{enumerate}
\subsubsection{课题评分表}

{\noindent
    
\begin{figure}[!hbtp]
\centering
\footnotesize
%\hspace{-3.4ex}
\renewcommand\arraystretch{1.9}
\begin{tabu}to 0.45\textwidth{|cc|c|c|c|c|c|c|}
\hline
\multicolumn{2}{|c|}{工件编号}&\multicolumn{2}{c}{}&
\multicolumn{2}{|c}{总得分}&\multicolumn{2}{|c|}{}\\
\hline
\multicolumn{2}{|c|}{项目与配分}&\parbox{2ex}{序号}&技术要求&配分
&评分标准&\parbox{4ex}{检测记录}&得分\\
\hline

\multicolumn{2}{|c|}{\multirow{3}{*}{\parbox{10ex}{程序管理(18\%)}}}&1&新建、删除、复制&6&操作不正确全扣&&\\
\cline{3-8}
&&2&检索、打开&6&操作不正确全扣&&\\ \cline{3-8}
&&3&程序信息显示&6&操作不正确全扣&&\\
\hline

\multicolumn{2}{|c|}{\multirow{1}{*}{\parbox{10ex}{程序录入(20\%)}}}&4&正确录入&20&出错一处扣2分&&\\[.1cm] \hline

\multicolumn{2}{|c|}{\multirow{3}{*}{\parbox{10ex}{程序编辑(15\%)}}}&5&光标定位、复位&5&操作不正确全扣&&\\  \cline{3-8}
&&6&字的检索、插入&5&操作不正确全扣&&\\ \cline{3-8}
&&7&字的删除、替换&5&操作不正确全扣&&\\ \hline

\multicolumn{2}{|c|}{\multirow{1}{*}{\parbox{10ex}{程序检查与模拟(10\%)}}}&8&程序检查与模拟&10&操作不正确全扣&&\\[.3cm] \hline

\multicolumn{2}{|c|}{\multirow{2}{*}{\parbox{10ex}{面板系统界面(15\%)}}}&9&操作面板&7&出错一处扣2分&&\\  \cline{3-8}
&&10&系统界面的认识&8&出错一处扣2分&&\\ \hline

\multicolumn{2}{|c|}{\multirow{4}{*}{\parbox{10ex}{机床操作(22\%)}}}&11&手动、手轮、快速&5&出错一处扣2分&&\\ \cline{3-8}
&&12&点的定位&7&出错一次扣2分&&\\ \cline{3-8}
&&13&机床操作规范&5&出错一次扣2分&&\\ \cline{3-8}
&&14&工件刀具装夹&5&出错一次扣2分&&\\ \hline

\multicolumn{2}{|c|}{\multirow{2}{*}{\parbox{10ex}{安全文明生产
(倒扣分)}
}}&15&安全操作&倒扣&
\multirow{2}{*}{\parbox{14ex}{安全事故停止操作或酌情扣分}}&&\\
\cline{3-5}\cline{7-8}
&&16&机床整理&倒扣&&&\\
\hline
\end{tabu}
\end{figure}}