\jxhj{%教学后记
	}
\skrq{%授课日期
	2017年5月15日 4-5节}
\ktmq{%课题名称
	 孔系变量编程}
\jxmb{%教学目标,每行前面要加 \item
	\item 掌握孔系的宏程序加工方法;
	\item 掌握孔系的编程思路;
	\item 掌握循环嵌套的使用;
	\item 分清Fanuc与Siemens的指令格式。 }
\jxzd{%教学重点,每行前面要加 \item
	\item 孔系的宏程序;
	\item 孔系的编程思路。 }
\jxnd{%教学难点,每行前面要加 \item
	\item 孔系的编程思路。 }
\jjff{%教学方法
	通过讲述、举例、演示法来说明;}

\makeshouye %制作教案首页

%%%%教学内容
\subsection{组织教学}
\begin{enumerate}[\hspace{2em}1、]
	\item 集中学生注意力;
	\item 清查学生人数;
	\item 维持课堂纪律;
\end{enumerate}
\subsection{复习导入及主要内容}
\begin{enumerate}[1、]
	\item 加工轮廓的处理;
	\item 极坐标;
	\item 加工工序。
\end{enumerate}


\subsection{教学内容及过程}


\subsubsection{在Fanuc上用G91+K来实现孔系加工}

\begin{lstlisting}[language=C]
O0001
G54G17G40G49G90
M3S500
G1Z30.F2000
X0Y0
G99G81X20.Y20.Z-20.R5.F80 K6
G1Z30.F2000
M5
M30
\end{lstlisting}

\subsubsection{宏程序来实现}

\begin{lstlisting}[language=C]
#24=  圆周圆心的X坐标绝对值
#25=  圆周圆心的Y坐标绝对值 
#26=  孔深Z坐标绝对值 
#18=  快速趋近点R坐标
#9=   切削进给速度F
#4=   圆半径1
#1=   第一孔的角度
#2=   增量角B
#11=  孔数H
G54G17G40G49G90
M3S800
G52 X#24 Y#25
G1 Z30.F2000
#8=1
WHILE[#8LE#11] DO1
#5=#4*COS[#1+[#8-1]*#2]
#5=#4*SIN[#1+[#8-1]*#2]
G99G81X#5Y#6Z#26R#18F#9
#8=#8+1
END1;
G1Z30.F2000
G52 X0 Y0
M5
M30
\end{lstlisting}


\subsubsection{方形阵列孔加工}

\begin{lstlisting}[language=C]
#1= 矩阵孔群横向中心连线与X轴的夹角
#2= 矩阵孔群横向中心与纵向中心连线角度
#3= 矩阵横向孔中心距
#4= 矩阵纵向孔中心距
#5= 矩阵横向孔数
#6= 矩阵纵向孔数
#9= 切削进给速度 Feed
#18= 固定循环中快速走近R点Z坐标
#24= 圆心X坐标
#25= 圆心Y坐标
#26= 孔深
G54G17G40G49G90
M3S500
G1Z30.F2000
G52 X#24 Y#25
G68 X0 Y0 R#1
#10=1
WHILE[#10LE#6]DO1
#11=1 
WHILE[#11LE#6]DO2
IF [[#10AND1]EQ0] GOTO1
#12=#3*[#11-1]+#4*COS[#2]*[#10-1]
#13=#4*SIN[#2]*[#10-1]
GOTO5
N1 #12=#3*[#5-#11]+#4*COS[#2]*[#10-1]
#13=#4*SIN[#2]*[#10-1]
N5 G99 G81 X#12 Y#13 Z#26 r#18 F#9
#11=#11+1
END2
#10=#10+1
END1
G80G1Z30.F2000
G69
G52 X0 Y0
M5
M30
\end{lstlisting}

\subsubsection{圆形阵列孔加工}

\begin{lstlisting}[language=C]
O0001
#1=40
#2=45
#3=8
#4=10
#5=5
S1000M3
G54G90  G1 X0 Y0 Z30
G16
#6=1
WHILE[#6LE#3]DO1
#7=1
WHILE[#7LE#5]DO2
#8=#1/2+[#7-1]*#4
#9=[#6-1]*#2
G99G81 X#8 Y#9 Z-6 R1 F80
#7=#7+1
END2
#6=#6+1
END1
G80G1Z30.F2000
G15
M5
M30
\end{lstlisting}

\subsubsection{混合孔的加工}

略



\subsection{课堂小结}
\begin{enumerate}[1、]
	\item 在Fanuc上用G91+K来实现孔系加工;
	\item 宏程序来实现;
	\item 方形阵列孔加工;
	\item 圆形阵列孔加工;
	\item 混合孔的加工。
\end{enumerate}

\vfill
\subsection{布置作业}
\begin{enumerate}[1、]
	\item 写出上面的程序;
	\item 从习题集上选做一个。
\end{enumerate}
\vfill