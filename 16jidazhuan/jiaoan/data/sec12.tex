\jxhj{%教学后记
	}
\skrq{%授课日期
	2017年10月24日 4-5节}
\ktmq{%课题名称
	 siemens编程应用}
\jxmb{%教学目标,每行前面要加 \item
	\item 掌握Siemens上子程序指令的使用;
	\item 掌握Siemens上的编程;
	\item 掌握Siemens上的半径补偿;
	\item 掌握相关注意事项。 }
\jxzd{%教学重点,每行前面要加 \item
	\item Siemens上的编程;
	\item Siemens上的半径补偿。 }
\jxnd{%教学难点,每行前面要加 \item
	\item Siemens上的半径补偿。 }
\jjff{%教学方法
	通过讲述、举例、演示法来说明;}

\makeshouye %制作教案首页

%%%%教学内容
\subsection{组织教学}
\begin{enumerate}[\hspace{2em}1、]
	\item 集中学生注意力;
	\item 清查学生人数;
	\item 维持课堂纪律;
\end{enumerate}
\subsection{复习导入及主要内容}
\begin{enumerate}[1、]
\item 子程序Z向分层;
\item 子程序XY向分层;
\item 子程序的应用。
\end{enumerate}

\subsection{教学内容及过程}

\subsubsection{Siemens上的程序名}
Fanuc: O+四位数值 为程序号,

即1号和0001号为同一个程序。

0-7999: 用户使用。不能加密。

8000-8999: 机床生产商和用户使用,可加密。

9000-9999: 系统开发商使用,可加密。

主程号与子程序号不区分。

Siemens:  前两位为字母,后面任意。不能与系统已有的字
重名。

GX03  AAA  BBB   可以

GOTO10  CYCLE81   不可以

字程序:   L+四位数值组成

也可以与主程序使用一样的规格。

主程序的扩展名为  MPF

子程序的扩展名为  SPF

\subsubsection{Siemens上的G指令}
1、G1    G2/G3

Siemens  使用CR=\_\_表示半径。

Siemens上还有其他使用格式。

2、G4  暂停时间

Fanuc: G4 X\_\_/P—  X表示秒,P表示毫秒。

Sienem: G4 F\_\_   F表示秒。

3、G17  G18  G19

4、G20  G21    fanuc上 表示 英制 与公制

G70  G71   Siemens上 表示 英制 与公制

5、G27 G28 G29  Fanuc 表现 回参考点检查  回参考点  从参考点返回。

G74  回参考点

6、G40  G41  G42

补偿号   在Siemens上称为刀沿号

每一把刀具可以有9个切削刀沿。

7、G43 G44  G49  Fanuc 刀具长度补偿

T1D1  Siemens 上自动补偿。

8、G53 G54-G59

\subsubsection{Siemens上其他问题}

1、子程序结束 RET M17

2、子程序的调用 

子程序名  P\_\_ 调用次数

子程序的调用必须单独占用一个程序段。

\subsubsection{Siemens编程实例}
如图: 在数控机床上加工如图所示的零件,试用siemens
格式编程。

程序:
\begin{lstlisting}
GX01   粗加工
G54G17G40G90
T1D1
M3S500
G1Z30.F2000
X-55.Y0
Z5.
Z0F200
L1 P4
Z0
L2 P2
Z3.
Z30.F2000
M5
M2	
\end{lstlisting}

\begin{lstlisting}
L1   方形轮廓子程序
G91Z-2.5
G90
G41G1X-45.Y0
G3X-35.Y0CR=10.
G1Y35.
X35.
Y-35.
X-35.
Y0
G3X-45.Y10.CR=10.
G40G1X-55.Y0
M17
\end{lstlisting}

\begin{lstlisting}
L2  圆形轮廓子程序
G91G1Z-2.5
G90
G41G1X-40.Y-10.
G3X-30.Y0CH=10.
G2I30.
G3X-40.Y10.CH=10.
G40G1X-55Y0
RET
\end{lstlisting}

\begin{lstlisting}
GX01   精加工
G54G17G40G90
T2D1
M3S800
G1Z30.F2000
X-55.Y0
Z5.
Z-7.6F100  
L1 
Z-2.5
L2 
Z3.
Z30.F2000
M5
M2
\end{lstlisting}

\subsection{课堂小结}
\begin{enumerate}[1、]
	\item Siemens上的程序名;
	\item Siemens上的G指令;
	\item Siemen上的子程序;
	\item Siemens编程实例。
\end{enumerate}

\vfill
\subsection{布置作业}
\begin{enumerate}[1、]
	\item 用Siemens格式编程一个程序。
\end{enumerate}
\vfill