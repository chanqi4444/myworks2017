
\documentclass{ctexart}

%%%%%%%%------------------------------------------------------------------------
%%%% 日常所用宏包

%% 控制页边距
\usepackage[top=2cm, bottom=2cm, left=3.2cm, right=2.cm,includehead,includefoot]{geometry}

%% 控制项目列表
\usepackage{enumerate}

%% 多栏显示
\usepackage{multicol}

%% hyperref宏包,生成可定位点击的超链接,并且会生成pdf书签
\usepackage[%
    pdfstartview=FitH,%
    CJKbookmarks=true,%
    bookmarks=true,%
    bookmarksnumbered=true,%
    bookmarksopen=true,%
    colorlinks=true,%
    citecolor=blue,%
    linkcolor=blue,%
    anchorcolor=green,%
    urlcolor=blue%
]{hyperref}

%% 控制标题
\usepackage{titlesec}

%% 控制表格样式
\usepackage{booktabs}

%% 控制目录
\usepackage{titletoc}

%% 控制字体大小
\usepackage{type1cm}

%% 首行缩进,用\noindent取消某段缩进
\usepackage{indentfirst}

%% 支持彩色文本、底色、文本框等
\usepackage{color,xcolor}

%% AMS LaTeX宏包
\usepackage{amsmath}

%% 一些特殊符号
% \usepackage{bbding}

%% 支持引用
% \usepackage{cite}

%% LaTeX一些特殊符号宏包
% \usepackage{latexsym}

%% 数学公式中的黑斜体
% \usepackage{bm}

%% 调整公式字体大小:\mathsmaller, \mathlarger
% \usepackage{relsize}

%% 生成索引
% \makeindex

%%%% 基本插图方法
%% 图形宏包
\usepackage{graphicx}

%% 多个图形并排,参加lnotes.pdf
\usepackage{subfig}

% \begin{figure}[htbp]               %% 控制插图位置
%   \setlength{\abovecaptionskip}{0pt}
%   \setlength{\belowcaptionskip}{10pt}
                                     %% 控制图形和上下文的距离
%   \centering                       %% 使图形居中显示
%   \includegraphics[width=0.8\textwidth]{CTeXLive2008.jpg}
                                     %% 控制图形显示宽度为0.8\textwidth
%   \caption{CTeXLive2008安装过程} \label{fig:CTeXLive2008}
                                     %% 图形题目和交叉引用标签
% \end{figure}
%%%% 基本插图方法结束

%%%% pgf/tikz绘图宏包设置
\usepackage{pgf,tikz}
\usetikzlibrary{shapes,automata,snakes,backgrounds,arrows}
\usetikzlibrary{mindmap}
%% 可以直接在latex文档中使用graphviz/dot语言,
%% 也可以用dot2tex工具将dot文件转换成tex文件再include进来
%% \usepackage[shell,pgf,outputdir={docgraphs/}]{dot2texi}
%%%% pgf/tikz设置结束


%%%% fancyhdr设置页眉页脚
%% 页眉页脚宏包
\usepackage{fancyhdr}

%% 页眉页脚风格
\pagestyle{plain}

%% 有时会出现\headheight too small的warning
\setlength{\headheight}{15pt}

%% 清空当前页眉页脚的默认设置
%\fancyhf{}
%%%% fancyhdr设置结束


%%%% 设置listings宏包用来粘贴源代码
%% 方便粘贴源代码,部分代码高亮功能
\usepackage{listings}

%% 所要粘贴代码的编程语言
\lstloadlanguages{}

%% 设置listings宏包的一些全局样式
%% 参考http://hi.baidu.com/shawpinlee/blog/item/9ec431cbae28e41cbe09e6e4.html
\lstset{
showstringspaces=false,              %% 设定是否显示代码之间的空格符号
numbers=left,                        %% 在左边显示行号
numberstyle=\tiny,                   %% 设定行号字体的大小
basicstyle=\footnotesize,                    %% 设定字体大小\tiny, \small, \Large等等
keywordstyle=\color{blue!70}, commentstyle=\color{red!50!green!50!blue!50},
                                     %% 关键字高亮
frame=shadowbox,                     %% 给代码加框
rulesepcolor=\color{red!20!green!20!blue!20},
escapechar=`,                        %% 中文逃逸字符,用于中英混排
xleftmargin=2em,xrightmargin=2em, aboveskip=1em,
breaklines,                          %% 这条命令可以让LaTeX自动将长的代码行换行排版
extendedchars=false                  %% 这一条命令可以解决代码跨页时,章节标题,页眉等汉字不显示的问题
}
%%%% listings宏包设置结束


%%%% 附录设置
\usepackage[title,titletoc,header]{appendix}
%%%% 附录设置结束


%%%% 日常宏包设置结束
%%%%%%%%------------------------------------------------------------------------

%%%%%%%%------------------------------------------------------------------------
%%%% 英文字体设置结束
%% 这里可以加入自己的英文字体设置
%%%%%%%%------------------------------------------------------------------------

%%%%%%%%------------------------------------------------------------------------
%%%% 设置常用字体字号,与MS Word相对应

%% 一号, 1.4倍行距
\newcommand{\yihao}{\fontsize{26pt}{36pt}\selectfont}
%% 二号, 1.25倍行距
\newcommand{\erhao}{\fontsize{22pt}{28pt}\selectfont}
%% 小二, 单倍行距
\newcommand{\xiaoer}{\fontsize{18pt}{18pt}\selectfont}
%% 三号, 1.5倍行距
\newcommand{\sanhao}{\fontsize{16pt}{24pt}\selectfont}
%% 小三, 1.5倍行距
\newcommand{\xiaosan}{\fontsize{15pt}{22pt}\selectfont}
%% 四号, 1.5倍行距
\newcommand{\sihao}{\fontsize{14pt}{21pt}\selectfont}
%% 半四, 1.5倍行距
\newcommand{\bansi}{\fontsize{13pt}{19.5pt}\selectfont}
%% 小四, 1.5倍行距
\newcommand{\xiaosi}{\fontsize{12pt}{18pt}\selectfont}
%% 大五, 单倍行距
\newcommand{\dawu}{\fontsize{11pt}{11pt}\selectfont}
%% 五号, 单倍行距
\newcommand{\wuhao}{\fontsize{10.5pt}{10.5pt}\selectfont}
%%%%%%%%------------------------------------------------------------------------


%%%%%%%%------------------------------------------------------------------------
%%%% 一些个性设置

%% 设定页码方式,包括arabic、roman等方式
%% \pagenumbering{arabic}

%% 有时LaTeX无从断行,产生overfull的错误,这条命令降低LaTeX断行标准
%% \sloppy

%% 设定目录显示深度\tableofcontents
%% \setcounter{tocdepth}{2}
%% 设定\listoftables显示深度
%% \setcounter{lotdepth}{2}
%% 设定\listoffigures显示深度
%% \setcounter{lofdepth}{2}

%% 设定段间距
\setlength{\parskip}{0.3\baselineskip}

%% 设定行距
\linespread{1}

%% 中文破折号,据说来自清华模板
\newcommand{\pozhehao}{\kern0.3ex\rule[0.8ex]{2em}{0.1ex}\kern0.3ex}

%% 设定itemize环境item的符号
\renewcommand{\labelitemi}{$\bullet$}

%% 设定正文字体大小
% \renewcommand{\normalsize}{\sihao}

%%%% 个性设置结束
%%%%%%%%------------------------------------------------------------------------


%%%%%%%%------------------------------------------------------------------------
%%%% bibtex设置

%% 设定参考文献显示风格
\bibliographystyle{unsrt}

%%%% bibtex设置结束
%%%%%%%%------------------------------------------------------------------------


%%%% 正文部分
\begin{document}
%%%%%%%%----------------------------------------------------
%%%% 开始首页
\xn{2017--2018} %学年
\xq{1} %学期
\xb{机电工程系} %系部
\zy{数控技术} %专业
\bj{16级大专数控班} %班级
\kc{《数控编程与实习》} %课程
\skzs{18} %上课周数
\zxs{4[3](3)} %周学时
\jk{×} %讲课
\sy{×} %实验
\lljk{66} %理论讲课
\sx{\zihao{5} (51)\par[51]} %实训
\sxlljk{×} %实习理论讲课
\scsx{×} %生产实习
\kh{4[3]} %考核
\jd{2(3)} %机动
\hj{\zihao{5}72\par(54)\par[54]} %合计
\ywcks{0} %已完成课时
\ylks{224}  %余留课时
\khfs{理论考试/实习考查}  %考核方式
\jcmc{数控机床编程与操作-数控铣床/加工中心分册~沈建峰}  %教材名称

\jhsy %生成计划首页
%%%% 结束首页
%%%%%%%%----------------------------------------------------

%%%%%%%%----------------------------------------------------
%%%% 计划说明
\begin{center}
\zihao{2} \heiti 学期授课计划说明
\end{center}
\zihao{-4} \setlength{\parindent}{2em} \setlength{\baselineskip}{22pt}

\textbf{一、教学目的与要求:}

本学期主要学习数控编程的基础知识,要求学生能熟练的掌握好数控编程的基础和一些简化指令的基本方法和思路去编程和解决实际问题,充分把自己 的能力及智慧通过编程展示出来。为后续的数控学习作好准备。

\textbf{二、用教材、参考书}

1、使用教材: 《数控机床编程与操作(数控铣床 加工中心分册)》 沈建峰

2、参考书:《加工中心编程与操作》  科学出版社  刘加孝   主编

\hspace{5em}《加工中心操作工》 中国劳动社会保障出版社  杨伟群  主编

\hspace{5em}《加工中心考工实训教程》  化学工业出版社   吴明友 主编

\textbf{三、教学措施}

1、采用多媒体、仿真、讨论等教学方法。

2、作业:理论课每周布置一道编程题,仿真每周做习题集上的题目,实习除了完成课题外,还要每个课题写一个实习报告。

3、学生评价采用自评、小组评价、教师评价三结合。

4、成绩平定,采用百分制,平时占70\%,包括出勤,作业,课堂答问等,期末闭卷占30\%。

\textbf{四、增删内容}

本计划无增删内容。

\textbf{五、本课程与其他课程的关系}

本课程是专业课,其他课程是基础,为本课服务。先要学习好《数控加工工艺》、《普铣》、《机械制图》、《机械加工原理》、《专业数学》等课程。在这些课程的基础上再来学习本课程就容易多了,希望同学们多复习这些课程。

\textbf{六、课程计划周数:}

授课时间为4--22周(第1周新生报道,第3周老生报道注册,第22周考试,放假1周),上课周数17周,周课时10节。

\onecolumn \setlength{\parindent}{0em}
%%%% 计划说明
%%%%%%%%----------------------------------------------------

%%%%%%%%----------------------------------------------------
%%%% 开始教学计划表
\begin{jxjhb}
1&新生报到、教师报到		& & & & & 08.15 08.20 & \\[6ex] \hline
2&新生上课、教师备课		& & & & & 08.21 08.27 & \\[6ex] \hline
3&老生报道、老生注册		& & & & & 08.28 09.03 & \\[6ex] \hline
4/1& 理论1、数控编程概术 & 掌握数控编程的基本知识&自绘示意图1&习题1& 2节 & 09.04 09.10 &  \\[6ex] \hline
4/2& 理论2、程序的基本结构 &掌握程序结构及了解基本指令 &自绘示意图2&习题2 & 2节 &  & \\[6ex] \hline

5/1& 理论3、基本指令及写程序思路 &掌握基本指令及编程基本思路 &自绘示意图3&习题3 &2节 & 09.11 09.17 & \\[6ex] \hline
5/2& 理论4、基本指令(一)	&掌握G0/G1指令及了解G2G3指令 &自绘示意图4&习题4 & 2节& & \\[6ex] \hline

6/1& 理论5、基本指令(二)	& 掌握G2/G3指令的两种用法&自绘示意图5 &习题5 &2节 & 09.18 09.24 & \\[6ex] \hline
6/2& 理论6、基本指令(三) & 掌握G2/G3指令的编程& 自绘示意图6&习题6 & 2节& & \\[6ex] \hline

7/1& 理论7、刀具半径补偿 &掌握用半径补偿指令编程 & 自绘示意图7&习题7 & 2节& 09.25 10.01 & \\[6ex] \hline
\end{jxjhb}

\begin{jxjhb}
7/2& 理论8、刀具半径补偿的应用(一)& 掌握用刀补进行加工精度的控制&自绘示意图8 &习题8 &2节&  & \\[6ex] \hline

8& 国庆放假		 & & & & & 10.02 10.08& \\[6ex] \hline

9/1& 理论9、刀具半径补偿的应用(二) & 掌握用刀补进行去残料& 自绘示意图9& 习题9& 2节& 10.09 10.15& \\[6ex] \hline		
9/2& 理论10、子程序概述及Z向分层 	& 掌握用子程序进行Z向分层&自绘示意图10 &习题10 &2节 &  & \\[6ex] \hline

10/1& 理论11、子程序的XY向分层 &掌握用子程序进行XY向分层& 自绘示意图11&习题11 &2节 & 10.16 10.22& \\[6ex] \hline
10/2& 理论12、siemens编程应用&掌握siemens上的程序及综合应用& 自绘示意图12&习题12& 2节& & \\[6ex] \hline

11/1& 理论13、平面加工工艺及编程 &掌握平面及斜面的加工编程 & 自绘示意图13&习题13 & 2节& 10.23 10.29& \\[6ex] \hline
11/2& 理论14、外轮廓加工(去残料) & 掌握用增加刀路去残料& 自绘示意图14&习题14 &2节 & & \\[6ex] \hline

12/1& 理论15、期中测试 &期中测试,了解学生掌握情况 &自绘示意图15 &习题15 & 2节& 10.30 11.05& \\[6ex] \hline
12/2& 理论16、试卷讲解 & 复习前半学期所学知识& 自绘示意图16&习题16& 2节& & \\[6ex] \hline

\end{jxjhb}

\begin{jxjhb}
	13/1& 理论17、岛屿型腔加工& 掌握岛屿槽的编程&自绘示意图17 &习题17 &2节&11.06 11.12  & \\[6ex] \hline
	13/2& 理论18、凸轮槽加工(正负刀补)		 & 掌握正负刀补的应用 &自绘示意图18 &习题18 &2节 &  & \\[6ex] \hline
	
	14/1& 理论19、孔加工概述 & 掌握孔加工工艺及编程思路& 自绘示意图19& 习题19& 2节& 11.13 11.19& \\[6ex] \hline		
	14/2& 理论20、Fanuc 上的孔加工循环(一) 	& 掌握Fanuc上孔加工指令&自绘示意图20 &习题20 &2节 &  & \\[6ex] \hline
	
	15/1& 理论21、Fanuc 上的孔加工循环(二) &掌握Fanuc孔的编程& 自绘示意图21&习题21 &2节 & 11.20 11.26& \\[6ex] \hline
	15/2& 理论22、Siemens 上的孔加工循环(一)&掌握Siemens上孔加工指令& 自绘示意图22&习题22& 2节& & \\[6ex] \hline
	
	16/1& 理论23、Siemens 上的孔加工循环(二) &会编Siemens上孔加工指令 & 自绘示意图23&习题23 & 2节& 11.27 12.03& \\[6ex] \hline
	16/2& 理论24、长度补偿概述 & 掌握长度补偿指令的使用& 自绘示意图24&习题24 &2节 & & \\[6ex] \hline
	
	17/1& 理论25、长度补偿的应用 &掌握用长度补偿来编程 &自绘示意图25 &习题25 & 2节& 12.04 12.10& \\[6ex] \hline
	17/2& 理论26、Siemens上的长度补偿 & 掌握Siemens上的长度补偿& 自绘示意图26&习题26& 2节& & \\[6ex] \hline
\end{jxjhb}

\begin{jxjhb}
	18/1& 理论27、加工中心编程& 掌握加工中心的编程技巧&自绘示意图27 &习题27 &2节&11.11 11.17  & \\[6ex] \hline
	18/2& 理论28、倒角与拐圆角 &掌握倒角与拐圆角使用 &自绘示意图28 &习题28 &2节 &  & \\[6ex] \hline
	
	19/1& 理论29、局部坐标系 & 掌握局部坐标系的使用& 自绘示意图29& 习题29& 2节& 11.18 11.24& \\[6ex] \hline		
	19/2& 理论30、极坐标编程 	& 掌握极坐标的编程&自绘示意图30 &习题30 &2节 &  & \\[6ex] \hline
	
	20/1& 理论31、坐标系旋转 &掌握坐标系旋转指令的使用& 自绘示意图31&习题31 &2节 & 11.25 11.31& \\[6ex] \hline
	20/2& 理论32、坐标系旋转(二)&掌握坐标系旋转指令的使用& 自绘示意图32&习题32& 2节& & \\[6ex] \hline
	
	21/1& 理论33、变量编程概述 &掌握变量编程的基本知识 & 自绘示意图33&习题33 & 2节& 01.01 01.07& \\[6ex] \hline
	21/2& 理论34、宏程序Z向分层 & 掌握宏程序Z向分层的编程& 自绘示意图34&习题34 &2节 & & \\[6ex] \hline
	
	22& 理论35、期末复习 &复习本学期所学知识 &自绘示意图35 &习题35 & 2节& 01.08 01.14& \\[6ex] \hline
	23&  期末考试、阅卷、成绩登录 & & && &01.15 01.21 & \\[6ex] \hline
\end{jxjhb}
\shqz %审核签字

\begin{jxjhb}
	1&新生报到、教师报到		& & & & & 08.15 08.20 & \\[6ex] \hline
	2&新生上课、教师备课		& & & & & 08.21 08.27 & \\[6ex] \hline
	3&老生报道、老生注册		& & & & & 08.28 09.03 & \\[6ex] \hline
	
	4& 实习1、安全操作及机床面板认识 &认识机床及掌握基本操作&数控机床及\par 相关工具 &实习报告1 & [3](3)& 09.04 09.10& \\[6ex] \hline
	
	5& 实习2、程序手工录入、编辑及刀路模拟 &掌握程序的快速录入与编辑 &数控机床及\par 相关工具 & 实习报告2& [3](3)& 09.11 09.17& \\[6ex] \hline
	
	6& 实习3、工件找正、装夹与对刀、调速 &掌握工件的对刀方法(取中法) &数控机床及\par 相关工具 &实习报告3 &  [3](3) & 09.18 09.24& \\[6ex] \hline

	7& 实习4、面铣及手动铣削 &会铣平面与简单零件加工 &数控机床及\par 相关工具 &实习报告4 &  [3](3) & 09.25 10.01& \\[6ex] \hline
	
	8& 国庆放假 & & & & & 10.02 10.08& \\[6ex] \hline
	
	9-12& 实习5、外轮廓加工 &掌握子程序的编程应用与精度控制\par 掌握增加刀路对零件去残料&数控机床及\par 相关工具 &实习报告5 &  [12](12)& 10.09 11.05& \\[6ex] \hline
	
	13-16& 实习6、型腔加工 &掌握挖槽加工工艺与精度控制\par 掌握岛屿类零件的编程与加工	&数控机床及\par 相关工具 &实习报告6 &  [12](12)& 11.06 12.03& \\[6ex] \hline
	
\end{jxjhb}


\begin{jxjhb}

	17-20& 实习7、综合加工(孔加工) &掌握孔的加工\par 	掌握综合结构零件的加工	&数控机床及\par 相关工具 &实习报告7 &  [12](12)& 12.04 12.31& \\[6ex] \hline

	21-22& 期末复习 &复习本学期所学知识 	&数控机床及\par 相关工具 &  &  [6](6)&  01.01 01.14& \\[6ex] \hline

	23&  期末考试、阅卷、成绩登录 & & && &01.15 01.21 & \\[6ex] \hline
	&  & & & & & & \\[6ex] \hline
	&  & & & & & & \\[6ex] \hline
	&  & & & & & & \\[6ex] \hline
	&  & & & & & & \\[6ex] \hline
	&  & & & & & & \\[6ex] \hline
	&  & & & & & & \\[6ex] \hline
	&  & & & & & & \\[6ex] \hline
\end{jxjhb}

\shqz %审核签字


%%%% 教学计划表结束
%%%%%%%%----------------------------------------------------

\end{document}
%%%% 正文部分结束
%%%%%%%%----------------------------------------------------