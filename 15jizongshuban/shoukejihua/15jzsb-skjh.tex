
\documentclass{ctexart}

%%%%%%%%------------------------------------------------------------------------
%%%% 日常所用宏包

%% 控制页边距
\usepackage[top=2cm, bottom=2cm, left=3.2cm, right=2.cm,includehead,includefoot]{geometry}

%% 控制项目列表
\usepackage{enumerate}

%% 多栏显示
\usepackage{multicol}

%% hyperref宏包,生成可定位点击的超链接,并且会生成pdf书签
\usepackage[%
    pdfstartview=FitH,%
    CJKbookmarks=true,%
    bookmarks=true,%
    bookmarksnumbered=true,%
    bookmarksopen=true,%
    colorlinks=true,%
    citecolor=blue,%
    linkcolor=blue,%
    anchorcolor=green,%
    urlcolor=blue%
]{hyperref}

%% 控制标题
\usepackage{titlesec}

%% 控制表格样式
\usepackage{booktabs}

%% 控制目录
\usepackage{titletoc}

%% 控制字体大小
\usepackage{type1cm}

%% 首行缩进,用\noindent取消某段缩进
\usepackage{indentfirst}

%% 支持彩色文本、底色、文本框等
\usepackage{color,xcolor}

%% AMS LaTeX宏包
\usepackage{amsmath}

%% 一些特殊符号
% \usepackage{bbding}

%% 支持引用
% \usepackage{cite}

%% LaTeX一些特殊符号宏包
% \usepackage{latexsym}

%% 数学公式中的黑斜体
% \usepackage{bm}

%% 调整公式字体大小:\mathsmaller, \mathlarger
% \usepackage{relsize}

%% 生成索引
% \makeindex

%%%% 基本插图方法
%% 图形宏包
\usepackage{graphicx}

%% 多个图形并排,参加lnotes.pdf
\usepackage{subfig}

% \begin{figure}[htbp]               %% 控制插图位置
%   \setlength{\abovecaptionskip}{0pt}
%   \setlength{\belowcaptionskip}{10pt}
                                     %% 控制图形和上下文的距离
%   \centering                       %% 使图形居中显示
%   \includegraphics[width=0.8\textwidth]{CTeXLive2008.jpg}
                                     %% 控制图形显示宽度为0.8\textwidth
%   \caption{CTeXLive2008安装过程} \label{fig:CTeXLive2008}
                                     %% 图形题目和交叉引用标签
% \end{figure}
%%%% 基本插图方法结束

%%%% pgf/tikz绘图宏包设置
\usepackage{pgf,tikz}
\usetikzlibrary{shapes,automata,snakes,backgrounds,arrows}
\usetikzlibrary{mindmap}
%% 可以直接在latex文档中使用graphviz/dot语言,
%% 也可以用dot2tex工具将dot文件转换成tex文件再include进来
%% \usepackage[shell,pgf,outputdir={docgraphs/}]{dot2texi}
%%%% pgf/tikz设置结束


%%%% fancyhdr设置页眉页脚
%% 页眉页脚宏包
\usepackage{fancyhdr}

%% 页眉页脚风格
\pagestyle{plain}

%% 有时会出现\headheight too small的warning
\setlength{\headheight}{15pt}

%% 清空当前页眉页脚的默认设置
%\fancyhf{}
%%%% fancyhdr设置结束


%%%% 设置listings宏包用来粘贴源代码
%% 方便粘贴源代码,部分代码高亮功能
\usepackage{listings}

%% 所要粘贴代码的编程语言
\lstloadlanguages{}

%% 设置listings宏包的一些全局样式
%% 参考http://hi.baidu.com/shawpinlee/blog/item/9ec431cbae28e41cbe09e6e4.html
\lstset{
showstringspaces=false,              %% 设定是否显示代码之间的空格符号
numbers=left,                        %% 在左边显示行号
numberstyle=\tiny,                   %% 设定行号字体的大小
basicstyle=\footnotesize,                    %% 设定字体大小\tiny, \small, \Large等等
keywordstyle=\color{blue!70}, commentstyle=\color{red!50!green!50!blue!50},
                                     %% 关键字高亮
frame=shadowbox,                     %% 给代码加框
rulesepcolor=\color{red!20!green!20!blue!20},
escapechar=`,                        %% 中文逃逸字符,用于中英混排
xleftmargin=2em,xrightmargin=2em, aboveskip=1em,
breaklines,                          %% 这条命令可以让LaTeX自动将长的代码行换行排版
extendedchars=false                  %% 这一条命令可以解决代码跨页时,章节标题,页眉等汉字不显示的问题
}
%%%% listings宏包设置结束


%%%% 附录设置
\usepackage[title,titletoc,header]{appendix}
%%%% 附录设置结束


%%%% 日常宏包设置结束
%%%%%%%%------------------------------------------------------------------------

%%%%%%%%------------------------------------------------------------------------
%%%% 英文字体设置结束
%% 这里可以加入自己的英文字体设置
%%%%%%%%------------------------------------------------------------------------

%%%%%%%%------------------------------------------------------------------------
%%%% 设置常用字体字号,与MS Word相对应

%% 一号, 1.4倍行距
\newcommand{\yihao}{\fontsize{26pt}{36pt}\selectfont}
%% 二号, 1.25倍行距
\newcommand{\erhao}{\fontsize{22pt}{28pt}\selectfont}
%% 小二, 单倍行距
\newcommand{\xiaoer}{\fontsize{18pt}{18pt}\selectfont}
%% 三号, 1.5倍行距
\newcommand{\sanhao}{\fontsize{16pt}{24pt}\selectfont}
%% 小三, 1.5倍行距
\newcommand{\xiaosan}{\fontsize{15pt}{22pt}\selectfont}
%% 四号, 1.5倍行距
\newcommand{\sihao}{\fontsize{14pt}{21pt}\selectfont}
%% 半四, 1.5倍行距
\newcommand{\bansi}{\fontsize{13pt}{19.5pt}\selectfont}
%% 小四, 1.5倍行距
\newcommand{\xiaosi}{\fontsize{12pt}{18pt}\selectfont}
%% 大五, 单倍行距
\newcommand{\dawu}{\fontsize{11pt}{11pt}\selectfont}
%% 五号, 单倍行距
\newcommand{\wuhao}{\fontsize{10.5pt}{10.5pt}\selectfont}
%%%%%%%%------------------------------------------------------------------------


%%%%%%%%------------------------------------------------------------------------
%%%% 一些个性设置

%% 设定页码方式,包括arabic、roman等方式
%% \pagenumbering{arabic}

%% 有时LaTeX无从断行,产生overfull的错误,这条命令降低LaTeX断行标准
%% \sloppy

%% 设定目录显示深度\tableofcontents
%% \setcounter{tocdepth}{2}
%% 设定\listoftables显示深度
%% \setcounter{lotdepth}{2}
%% 设定\listoffigures显示深度
%% \setcounter{lofdepth}{2}

%% 设定段间距
\setlength{\parskip}{0.3\baselineskip}

%% 设定行距
\linespread{1}

%% 中文破折号,据说来自清华模板
\newcommand{\pozhehao}{\kern0.3ex\rule[0.8ex]{2em}{0.1ex}\kern0.3ex}

%% 设定itemize环境item的符号
\renewcommand{\labelitemi}{$\bullet$}

%% 设定正文字体大小
% \renewcommand{\normalsize}{\sihao}

%%%% 个性设置结束
%%%%%%%%------------------------------------------------------------------------


%%%%%%%%------------------------------------------------------------------------
%%%% bibtex设置

%% 设定参考文献显示风格
\bibliographystyle{unsrt}

%%%% bibtex设置结束
%%%%%%%%------------------------------------------------------------------------


%%%% 正文部分
\begin{document}
%%%%%%%%----------------------------------------------------
%%%% 开始首页
\xn{2017--2018} %学年
\xq{1} %学期
\xb{机电工程系} %系部
\zy{数控技术} %专业
\bj{15级中数控班} %班级
\kc{《数控编程与实习》} %课程
\skzs{18} %上课周数
\zxs{[3](3)} %周学时
\jk{×} %讲课
\sy{×} %实验
\lljk{×} %理论讲课
\sx{(51)\par[51]} %实训
\sxlljk{×} %实习理论讲课
\scsx{×} %生产实习
\kh{[3]} %考核
\jd{(3)} %机动
\hj{(54)\par[54]} %合计
\ywcks{240} %已完成课时
\ylks{0}  %余留课时
\khfs{实习考查}  %考核方式
\jcmc{数控机床编程与操作-数控铣床/加工中心分册~沈建峰}  %教材名称

\jhsy %生成计划首页
%%%% 结束首页
%%%%%%%%----------------------------------------------------

%%%%%%%%----------------------------------------------------
%%%% 计划说明
\begin{center}
\zihao{2} \heiti 学期授课计划说明
\end{center}
\zihao{-4} \setlength{\parindent}{2em} \setlength{\baselineskip}{22pt}

\textbf{一、教学目的与要求:}

本课程为专业核心课程,本学期通过自动编程入门的学习掌握加工中心的编程,通过平面类零件、曲面类零件、多面零件、综合零件的加工进一步掌握加工中心机床的操作,提高机床操作的熟练度,提高零件的加工精度。

\textbf{二、用教材、参考书}

1、使用教材: 《数控机床编程与操作(数控铣床 加工中心分册)》 沈建峰

2、参考书:《加工中心编程与操作》  科学出版社  刘加孝   主编

\hspace{5em}《加工中心操作工》 中国劳动社会保障出版社  杨伟群  主编

\hspace{5em}《加工中心考工实训教程》  化学工业出版社   吴明友 主编

\textbf{三、教学措施}

1、采用多媒体、仿真、讨论等教学方法。

2、作业:仿真每周做习题集上的题目,实习除了完成课题外,还要每个课题写一个实习报告。

3、学生评价采用自评、小组评价、教师评价三结合。

%4、成绩平定,采用百分制,平时占70\%,包括出勤,作业,课堂答问等,期末测试占30\%。

\textbf{四、增删内容}

本计划无增删内容。

\textbf{五、本课程与其他课程的关系}

本课程是专业课,其他课程是基础,为本课服务。先要学习好《数控加工工艺》、《普铣》、《机械制图》、《机械加工原理》、《专业数学》等课程。在这些课程的基础上再来学习本课程就容易多了,希望同学们多复习这些课程。

\textbf{六、课程计划周数:}

授课时间为4--22周(第1周新生报道,第3周老生报道注册,第22周考试,放假1周),上课周数17周,周课时10节。

\onecolumn \setlength{\parindent}{0em}
%%%% 计划说明
%%%%%%%%----------------------------------------------------

%%%%%%%%----------------------------------------------------
%%%% 开始教学计划表
\begin{jxjhb}
	1&新生报到、教师报到		& & & & & 08.15 08.20 & \\[6ex] \hline
	2&新生上课、教师备课		& & & & & 08.21 08.27 & \\[6ex] \hline
	3&老生报道、老生注册		& & & & & 08.28 09.03 & \\[6ex] \hline
	
	4-7& 产品1、平面类零件自动加工 &掌握平面类零件自动加工\par 掌握平面类零件刀路生成 &数控机床及\par 相关工具 &实习报告1 & [12](12)& 09.04 10.01& \\[6ex] \hline

	8& 国庆放假 & & & & & 10.02 10.08& \\[6ex] \hline
	
	9-12& 产品2、曲面类零件自动加工 &掌握曲面类零件自动加工\par 掌握曲面类零件刀路生成  &数控机床及\par 相关工具 & 实习报告2& [12](12)& 10.09 11.05& \\[6ex] \hline
	
	13-16& 产品3、多面加工零件加工 &掌握多次装夹内零件的加工\par 掌握自动编程的灵活运用 &数控机床及\par 相关工具 &实习报告3 &  [12](12) & 11.06 12.03& \\[6ex] \hline

	17-20& 产品4、综合零件加工 &会自己制定零件的加工工艺 &数控机床及\par 相关工具 &实习报告4 &  [12](12) & 12.04 12.31& \\[6ex] \hline
	

	21-22& 期末复习测试 &复习本学期所学知识 	&数控机床及\par 相关工具 &  &  [6](6)&  01.01 01.14& \\[6ex] \hline

	23&  期末考试、阅卷、成绩登录 & & && &01.15 01.21 & \\[6ex] \hline
	
%	&  & & & & & & \\[6ex] \hline	
	
\end{jxjhb}

\shqz %审核签字


%%%% 教学计划表结束
%%%%%%%%----------------------------------------------------

\end{document}
%%%% 正文部分结束
%%%%%%%%----------------------------------------------------