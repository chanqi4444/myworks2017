\documentclass[a4paper,zihao=-4,linespread=1.35]{ctexart}
%设定章节格式
\ctexset{
section/name={,、\hspace{-1em}},
section/number=\chinese{section},
section/format=\zihao{4}\sffamily\heiti\raggedright,
section/indent=28pt,
section/fixskip = true,
subsection/name={,、\hspace{-1em}},
subsection/number=\arabic{subsection},
subsection/format=\zihao{4}\sffamily\songti\raggedright,
subsection/indent=28pt,
subsection/fixskip = true,
%punct = kaiming,
}	
%设定纸张大小
\usepackage[top=2.54cm, bottom=2.54cm, left=3.17cm, right=3.17cm]{geometry}
%设定页眉页脚
\usepackage{fancyhdr}
\pagestyle{fancy}
\fancyhead{}
\lhead{} \rhead{} \chead{}
\cfoot{-\ \thepage\ -}
\renewcommand{\headrulewidth}{0pt}

%列表设定
\usepackage{enumerate} 
\usepackage{paralist} 
\let\itemize\compactitem 
\let\enditemize\endcompactitem 
\let\enumerate\compactenum 
\let\endenumerate\endcompactenum 
\let\description\compactdesc 
\let\enddescription\endcompactdesc

\usepackage{tabu}
\usepackage{multirow} 

\usepackage[colorlinks,
linkcolor=blue,
anchorcolor=blue,
citecolor=green
]{hyperref}

%正文部分
\begin{document}
\zihao{4}%设定正文字号4号
\begin{center}
\songti \bf \zihao{3}机电工程系参加2017年 湖南省职业院校技能竞赛 \par
工业产品数字化设计与制造\ \ 项目比赛的培训方案
\end{center} \par
根据学院安排,响应学院“以竞赛促技能提高、以竞赛促教学改革”的号召,
组队报名参加2017年湖南省职业院校技能竞赛工业产品数字化设计与制造项目比赛。
为使本次比赛取得较好成绩,培养下批参赛选手,特制定本培训方案。

\section{指导老师安排}
系部成立竞赛指导教师小组,成员包括刘加孝、高星、李云义、姚永辉,
成员之间分工合作:
李云义主要指导3D数据采集、点云处理;
高星主要指导UG逆向造型、UG数控编程、机床操作;
刘加孝主要指导创新设计及整体流程;
姚永辉整体协调安排。

\section{选手选拔安排}

\subsection{学生报名方式}
选手从14级5年大专、16级大专里选取,(选手可参加明年的比赛)
自愿报名,同时要求数控模具教研室根据平时教学中的学生表现推荐学生报名。

\subsection{入围选手选拔}

报名学生根据培训安排参加第一阶段培训,
指导教师负责对报名学生进行考核,
按不低于1:2的比例确定入围选手名单。

\subsection{确定参赛选手名单}
入围选手参加第二阶段培训,
指导教师负责对入围选手进行考核,
在学院正式报名期限内确定最终参加比赛选手名单(2名)。  

\section{选手培训安排}
\subsection{第一阶段}
时间:3月11日至3月31日 学生停课集中培训。\par
地点:学院实训中心一楼数控车间、三楼18机房。\par
要求:确定入围选手名单,并使选手对竞赛内容认识完整,掌握全面。\par

\subsection{第二阶段}

时间:4月1日至4月5日 模拟比赛。\par
地点:学院实训中心一楼数控车间、三楼18机房。\par
要求:确定参赛选手名单,并按比赛要求,进行针对性训练。\par

\subsection{第三阶段}

时间:4月6日至竞赛日 考前强化培训。
地点:学院实训中心一楼数控车间、三楼18机房。
要求:按比赛要求,根据选手特长、确定分工,培养团队合作。

\section{培训内容}
由于时间短,必须围绕比赛核心内容,学习必要的理论,加大实操训练。
提高学生的速度。\par

\begin{enumerate}[\indent1、]
	\item 三维天下扫描仪的使用。
	\item Win3d扫描软件的使用。
	\item Geomagic Design X 2016 点云数据处理。
	\item Geomagic Design X 2016 特征提取。
	\item Siemens NX 10.0 基本操作。
	\item Siemens NX 正向设计。
	\item  Siemens NX 逆向设计。
	\item Siemens NX 数控加工。
	\item 数控机床的基本操作、自动加工零件。
\end{enumerate}

\section{培训环境}

\subsection{硬件环境}
为确保参加竞赛及完善实习、实训场地急需增购如表 \ref{购物清单}\/:
\begin{table}[ht]
\begin{center}\zihao{5}

		\begin{tabular}{|c|c|c|c|c|c|c|}  
		\hline 
编号	&品名	&型号	&单位	&单价	&数量	&金额\\\hline
1	&6061硬铝	&150*200	&块	&5	&100.00元	&500\\\hline
2	&D16铣刀杆	&D16	&个	&80	&10	&800.00元\\\hline
3	&铣刀片	&D16配套	&盒	&150	&3	&450.00元\\\hline
4	&角磨机	&通用	&个	&150	&5	&750.00元\\\hline
5	&金刚石砂轮	&通用	&个	&100	&3	&300.00元\\\hline
6	&合金铣刀	&D10	&个	&180	&10	&1000.00元\\\hline
7	&合金铣刀	&D12	&个	&250	&5	&750.00元\\\hline
8	&其他小配件	&	&	&	&	&500.00元\\\hline
&合计	&	&	&	&	&5050\\\hline
	\end{tabular}  		\caption{购物清单}  \label{购物清单}
\end{center}
\end{table} 

由于学校没有3D扫描仪,计划外出培训,费用以实用另计。

\subsection{软件环境}
\begin{enumerate}[\indent A、]
\item Geomagic Design X 2016 ;
\item Siemens NX 10.0 基本操作;
\end{enumerate}

\section{其他事项} 
\begin{enumerate}[\indent1、]
	\item 培训场地准备事项  责任人:高星
\par
	负责搭建培训所需的软硬件环境、提供培训所需的设备或耗材。 
	\item 培训期间的协调工作 责任人:姚永辉
\par
	责培训期间安排指导老师、选拔参赛选手以及协调培训过程中的各项工作。 
	\item 培训考勤管理 责任人:姚永辉
\par
	负责培训期间严格按不同培训阶段要求对学生进行考勤管理。 
	\item 奖罚措施
\par
	执行《学院专业技能竞赛管理办法》文件精神。
\end{enumerate}
\section{具体时间安排} 
见表 \ref{具体时间安排}\/。\par
一阶段课时合计$22\times5=110$节。\par
后续培训可根据省里的文件动态调整。


\begin{table}[ht]
	\begin{center}\zihao{4}

		\begin{tabular}{|c|c|c|l|l|c|c|}  
			\hline  
			\multirow{2}{*}{序号} 	&\multirow{2}{*}{日期	}&\multirow{2}{*}{星期}	&\multicolumn{2}{c|}{指导教师}	&\multirow{2}{*}{场地}	&\multirow{2}{*}{内容	}\\ \cline{4-5}
				& & &	上午	&下午	&&	\\\hline
			1	&3.10	&五	&无	&全体	&办公室	&动员,介绍	\\\hline
			2	&3.11	&六	&高星	&高星	&18机房	&点云处理	\\\hline
			3	&3.12	&日	&李云义	&李云义	&18机房	&点云处理	\\\hline
			4	&3.13	&一	&高星	&刘加孝	&18机房	&点云处理	\\\hline
			5	&3.14	&二	&李云义	&刘加孝	&18机房	&UG基本操作	\\\hline
			6	&3.15	&三	&高星	&高星	&18机房	&UG基本操作	\\\hline
			7	&3.16	&四	&李云义	&李云义	&18机房	&UG基本操作	\\\hline
			8	&3.17	&五	&李云义	&高星	&18机房	&UG逆向造型	\\\hline
			9	&3.18	&六	&高星	&高星	&18机房	&UG逆向造型	\\\hline
			10	&3.19	&日	&李云义	&李云义	&18机房	&UG逆向造型	\\\hline
			11	&3.20	&一	&高星	&刘加孝	&18机房	&UG逆向造型	\\\hline
			12	&3.21	&二	&李云义	&刘加孝	&18机房	&UG创新设计	\\\hline
			13	&3.22	&三	&高星	&高星	&18机房	&数控机床操作	\\\hline
			14	&3.23	&四	&李云义	&李云义	&数控车间	&UG数控编程	\\\hline
			15	&3.24	&五	&李云义	&高星	&数控车间	&UG数控编程	\\\hline
			16	&3.25	&六	&高星	&高星	&数控车间	&UG数控编程	\\\hline
			17	&3.26	&日	&李云义	&李云义	&数控车间	&数控机床操作	\\\hline
			18	&3.27	&一	&高星	&刘加孝	&数控车间	&数控机床操作	\\\hline
			19	&3.28	&二	&李云义	&刘加孝	&数控车间	&数控机床操作	\\\hline
			20	&3.29	&三	&高星	&高星	&数控车间	&数控机床操作	\\\hline
			21	&3.30	&四	&李云义	&李云义	&	&3D扫描机使用	\\\hline
			22	&3.31	&五	&李云义	&高星	&	&3D扫描机使用	\\ \hline
			23	&3.10-3.31	&	\multicolumn{3}{c|}{姚永辉}	&	&考勤	\\ \hline
 	\end{tabular}  		
 \caption{具体时间安排}  \label{具体时间安排}
\end{center}
\end{table} 			

\hspace{10.4cm}机电工程系 \par
\hspace{9.6cm}2017年3月10日
%\newpage
\end{document}